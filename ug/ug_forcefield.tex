\section{Force Field Parameters}
\label{section:forcefield}


\subsection{Potential energy functions}

Evaluating the force is the most computationally demanding
part of molecular dynamics.
The force is the negative gradient of a scalar potential energy function,
\begin{equation}
\vec{F}(\vec{r}) = -\nabla U(\vec{r}),
\end{equation}
and, for systems of biomolecules,
this potential function involves the summing,
\begin{equation}
U(\vec{r}) = \sum U_{\text{bonded}}(\vec{r})
  + \sum U_{\text{nonbonded}}(\vec{r}),
\end{equation}
over a large number of bonded and nonbonded terms.
The bonded potential terms involve 2--, 3--, and 4--body interactions
of covalently bonded atoms,
with $O(N)$ terms in the summation.
The nonbonded potential terms involve interactions
between all pairs of atoms
(usually excluding pairs of atoms already involved in a bonded term),
with $O(N^2)$ terms in the summation,
although fast evaluation techniques are used to
compute good approximations to their contribution to the potential
with $O(N)$ or $O(N \log N)$ computational cost.


\subsubsection{Bonded potential energy terms}
%\label{sec:bonded}

The bonded potential terms involve 2--, 3--, and 4--body interactions
of covalently bonded atoms.

The 2--body spring bond potential
describes the harmonic vibrational motion
between an $(i,j)$--pair of covalently bonded atoms,
\begin{equation}
U_{\text{bond}} = k (r_{ij} - r_0)^2,
\end{equation}
where $r_{ij} = \|\vec{r}_j - \vec{r}_i\|$ gives the distance
between the atoms,
$r_0$ is the equilibrium distance,
and $k$ is the spring constant.

The 3--body angular bond potential
describes the angular vibrational motion
occurring between an $(i,j,k)$--triple of covalently bonded atoms,
\begin{equation}
U_{\text{angle}} = k_{\theta} (\theta - \theta_0)^2
                 + k_{\text{ub}} (r_{ik} - r_{\text{ub}})^2,
\end{equation}
where, in the first term,
$\theta$ is the angle in radians between vectors
$\vec{r}_{ij} = \vec{r}_j - \vec{r}_i$
and $\vec{r}_{kj} = \vec{r}_j - \vec{r}_k$,
$\theta_0$ is the equilibrium angle,
and $k_{\theta}$ is the angle constant.
The second term is the Urey--Bradley term
used to describe a 
(noncovalent) spring between the outer $i$ and $k$ atoms,
active when constant $k_{\text{ub}} \neq 0$,
where, like the spring bond,
$r_{ik} = \|\vec{r}_k - \vec{r}_i\|$ gives the distance between
the pair of atoms and
$r_{\text{ub}}$ is the equilibrium distance.

The 4--body torsion angle (also known as dihedral angle) potential
describes the angular spring between the planes formed
by the first three and last three atoms of
a consecutively bonded $(i,j,k,l)$--quadruple of atoms,
\begin{equation}
U_{\text{tors}} =
  \begin{cases}
  k (1 + \cos(n \psi + \phi)) & \text{if $n > 0$,} \\
  k (\psi - \phi)^2           & \text{if $n = 0$,}
  \end{cases}
\end{equation}
where $\psi$ is the angle in radians between
the $(i,j,k)$--plane and the $(j,k,l)$--plane.
The integer constant $n$ is nonnegative and indicates the periodicity.
For $n > 0$, $\phi$ is the phase shift angle
and $k$ is the multiplicative constant.
For $n = 0$, $\phi$ acts as an equilibrium angle
and the units of $k$ change to $\text{potential}/\text{rad}^2$.
A given $(i,j,k,l)$--quadruple of atoms might contribute
multiple terms to the potential,
each with its own parameterization.
The use of multiple terms for a torsion angle allows for
complex angular variation of the potential,
effectively a truncated Fourier series.


\subsubsection{Nonbonded potential energy terms}
%\label{sec:nonbonded}

The nonbonded potential terms involve interactions
between all $(i,j)$--pairs of atoms,
usually excluding pairs of atoms already involved in a bonded term.
Even using a fast evaluation methods
the cost of computing the nonbonded potentials dominates the work
required for each time step of an MD simulation.

The Lennard--Jones potential
accounts for the weak dipole attraction between distant atoms and
the hard core repulsion as atoms become close,
\begin{equation}
U_{\text{LJ}} = (-E_{\text{min}}) \left[ 
    \left( \frac{R_{\text{min}}}{r_{ij}} \right)^{12} -
    2 \left( \frac{R_{\text{min}}}{r_{ij}} \right)^{6} \right],
\end{equation}
where $r_{ij} = \|\vec{r}_j - \vec{r}_i\|$ gives the distance
between the pair of atoms.
The parameter $E_{\text{min}} = U_{\text{LJ}}(R_{\text{min}})$ is 
the minimum of the potential term
($E_{\text{min}} < 0$, which means that $-E_{\text{min}}$ is the well-depth).
The Lennard--Jones potential approaches 0 rapidly as $r_{ij}$
increases, so it is usually truncated (smoothly shifted) to 0
past a cutoff radius, requiring $O(N)$ computational cost.

The electrostatic potential
is repulsive for atomic charges with the same sign
and attractive for atomic charges with opposite signs,
\begin{equation}
U_{\text{elec}} = \epsilon_{14} \frac{C q_i q_j}{\epsilon_0 r_{ij}},
\end{equation}
where $r_{ij} = \|\vec{r}_j - \vec{r}_i\|$ gives the distance
between the pair of atoms,
and $q_i$ and $q_j$ are the charges on the respective atoms.
Coulomb's constant $C$ and the dielectric constant $\epsilon_0$
are fixed for all electrostatic interactions.
The parameter $\epsilon_{14}$ is a unitless scaling factor 
whose value is 1,
except for a modified 1--4 interaction,
where the pair of atoms is separated by a sequence
of three covalent bonds (so that the atoms might
also be involved in a torsion angle interaction),
in which case $\epsilon_{14} = \varepsilon$,
for a fixed constant $0 \leq \varepsilon \leq 1$.
Although the electrostatic potential may be computed with
a cutoff like the Lennard--Jones potential,
the $1/r$ potential approaches 0 much more
slowly than the $1/r^6$ potential,
so neglecting the long range electrostatic terms
can degrade qualitative results,
especially for highly charged systems.
There are other fast evaluation methods that approximate
the contribution to the long range electrostatic terms
that require $O(N)$ or $O(N \log N)$ computational cost,
depending on the method.


\subsection{Non-bonded interactions}
\label{section:electdesc}

\NAMD\ has a number of options that control the way that non-bonded
interactions are calculated.  These options are interrelated and
can be quite confusing, so this section attempts to explain the
behavior of the non-bonded interactions and how to use these
parameters.

\subsubsection{Van der Waals interactions}
The simplest non-bonded 
interaction is the van der Waals interaction.  In 
\NAMD, van der Waals interactions are always truncated at the 
cutoff distance, specified by {\tt cutoff}.  
The main option that effects van der Waals interactions
is the {\tt switching} parameter.  With this option set to {\tt on},
a smooth switching function will be used to truncate the
van der Waals potential energy smoothly at the cutoff distance.  
A graph of the van der Waals 
potential with this switching function is shown in Figure 
\ref{fig:switching}.  If {\tt switching} is set to {\tt off}, the 
van der Waals energy is just abruptly truncated at the cutoff 
distance, so that energy may not be conserved.  

\begin{figure}[htb]
  \center{\includegraphics{figures/switching}}
  \caption[Graph of van der Waals potential with and without switching]
  {\small Graph of van der Waals potential with and without the
  application of the switching function.  With the switching function
  active, the potential is smoothly reduced to 0 at the cutoff distance.
  Without the switching function, there is a discontinuity where the
  potential is truncated.}
  \label{fig:switching}
\end{figure}

The switching function used is based on the X-PLOR switching
function.  The parameter {\tt switchdist} specifies the distance
at which the switching function should start taking effect to
bring the van der Waals potential to 0 smoothly at the cutoff distance.  
Thus, the value of {\tt switchdist} must always be less than that 
of {\tt cutoff}.


\subsubsection{Electrostatic interactions}
The handling of electrostatics is slightly
more complicated due to the incorporation of multiple timestepping for full
electrostatic interactions.  There are two cases to consider, one where
full electrostatics is employed and the other where electrostatics
are truncated at a given distance.
\prettypar
First let us consider the latter case, where electrostatics are truncated at
the cutoff distance.  Using this scheme, all electrostatic interactions
beyond a specified distance are ignored, or assumed to be zero.  If
{\tt switching} is set to {\tt on}, rather than having a discontinuity
in the potential
at the cutoff distance, a shifting function is applied to the electrostatic
potential as shown in Figure \ref{fig:shifting}.  As this figure shows, the
shifting function shifts the entire potential curve so that the curve
intersects the x-axis at the cutoff distance.  This shifting function
is based on the
shifting function used by X-PLOR.

\begin{figure}[htb]
  \center{\includegraphics{figures/shifting}}
  \caption[Graph of electrostatic potential with and without shifting function]
  {\small Graph showing an electrostatic potential with and without the
  application of the shifting function.}
  \label{fig:shifting}
\end{figure}

Next, consider the case where full electrostatics are calculated.  In this
case, the electrostatic interactions are not truncated at any distance.  In
this scheme, the {\tt cutoff} parameter has a slightly different meaning
for the electrostatic interactions --- it represents
the {\it local interaction distance\/}, or distance within which electrostatic
pairs will be directly calculated every timestep.  Outside of this distance,
interactions will be calculated only periodically.  These forces
will be applied using a multiple timestep integration scheme as described in
Section \ref{section:mts}.

\begin{figure}[htb]
  \center{\includegraphics{figures/fmaOn}}
  \caption[Graph of electrostatic split between short and long range forces]
  {\small Graph showing an electrostatic potential 
  when full electrostatics are used within \NAMD, 
  with one curve portion calculated directly 
  and the other calculated using DPMTA.}
  \label{fig:fmaOn}
\end{figure}



\subsubsection{Non-bonded force field parameters}

\begin{itemize}
%\item
%\NAMDCONF{eleccutoff}%
%{local interaction distance for electrostatic calculations (\AA)}%
%{positive decimal}%
%{If DPMTA is active, this distance defines the local interaction length
%for the DPMTA algorithm.
%Otherwise, this value specifies the distance at which
%electrostatic interactions are truncated.
%If {\tt eleccutoff} is defined, it supersedes {\tt cutoff}.
%If {\tt eleccutoff} is not defined, then \verb }cutoff} {\em must}
%be defined.
%See Section \ref{section:electdesc} for a further discussion
%of this configuration value.}

%\item
%\NAMDCONF{vdwcutoff}%
%{local interaction distance for van der Waals calculations (\AA)}%
%{positive decimal}%
%{This value specifies the distance at which
%van der Waals interactions are truncated.
%If {\tt vdwcutoff} is defined, it supersedes {\tt cutoff}.
%If {\tt vdwcutoff} is not defined, then \verb }cutoff} {\em must}
%be defined.
%See Section \ref{section:electdesc} for a further discussion
%of this configuration value.}

\item
\NAMDCONF{cutoff}
{local interaction distance common to both electrostatic 
and van der Waals calculations (\AA)}
{positive decimal}
{%This value can substitute for either {\tt eleccutoff}
%or {\tt vdwcutoff} if either of those is undefined.
See Section \ref{section:electdesc} for more information.}

\item
\NAMDCONFWDEF{switching}{use switching function?}{{\tt on} or {\tt off}}
{{\tt on}}
{If {\tt switching} is
specified to be {\tt off}, then a truncated cutoff is performed.
If {\tt switching} is turned {\tt on}, then smoothing functions
are applied to both the electrostatics and van der Waals forces.
For a complete description of the non-bonded force parameters see
Section \ref{section:electdesc}.  If {\tt switching} is set to
{\tt on}, then {\tt switchdist} must also be defined.}

\item
\NAMDCONFWDEF{vdwForceSwitching}{use force switching for VDW?}{{\tt on} or {\tt off}}
{{\tt off}}
{If both {\tt switching} and {\tt vdwForceSwitching} are set to {\tt on},
then CHARMM force switching is used for van der Waals forces.
{\bf LJcorrection as implemented is inconsistent with vdwForceSwitching.}}

%\item
%\NAMDCONF{elecswitchdist}{distance at which to activate switching function for %electrostatic calculations (\AA)}{positive decimal $\leq$ {\tt eleccutoff}}
%{Distance at which the switching function
%used to smooth the truncation of
%electrostatic forces should begin to take effect.  
%This parameter only has meaning if {\tt switching} is 
%set to {\tt on}.  
%The value of {\tt elecswitchdist} must be less than
%or equal to the value of {\tt eleccutoff}, since the switching function
%is only applied on the range from {\tt elecswitchdist} to {\tt eleccutoff}.
%If {\tt elecswitchdist} is defined, it supersedes {\tt switchdist}.
%If {\tt elecswitchdist} is not defined and {\tt switching} is
%{\tt on}, then \verb }switchdist} {\em must} be defined.
%For a complete description of the non-bonded force parameters, see
%Section \ref{section:electdesc}.
%}

%\item
%\NAMDCONF{vdwswitchdist}%
%{distance at which to activate switching function 
%for van der Waals calculations (\AA)}%
%{positive decimal $\leq$ {\tt vdwcutoff}}%
%{Distance at which the switching function
%used to smooth the truncation of
%van der Waals forces should begin to take effect.  
%This parameter only has meaning if {\tt switching} is 
%set to {\tt on}.  
%The value of {\tt vdwswitchdist} must be less than
%or equal to the value of {\tt vdwcutoff}, since the switching function
%is only applied on the range from {\tt vdwswitchdist} to {\tt vdwcutoff}.
%If {\tt vdwswitchdist} is defined, it supersedes {\tt switchdist}.
%If {\tt vdwswitchdist} is not defined and {\tt switching} is
%{\tt on}, then \verb }switchdist} {\em must} be defined.
%For a complete description of the non-bonded force parameters, see
%Section \ref{section:electdesc}.
%}

\item
\NAMDCONF{switchdist}
{distance at which to activate switching/splitting function 
for electrostatic and van der Waals calculations (\AA)}
{positive decimal $\leq$ {\tt cutoff}}
{Distance at which the switching function
should begin to take effect.  
This parameter only has meaning if {\tt switching} is 
set to {\tt on}.  
The value of {\tt switchdist} must be less than
or equal to the value of {\tt cutoff}, since the switching function
is only applied on the range from {\tt switchdist} to {\tt cutoff}.  
For a complete description of the non-bonded force parameters see
Section \ref{section:electdesc}.}

\item
\NAMDCONF{exclude}
{non-bonded exclusion policy to use}
{{\tt none}, {\tt 1-2}, {\tt 1-3}, {\tt 1-4}, or {\tt scaled1-4}}
{\label{param:exclude}
%% This parameter is {\it required\/} for every simulation.
This parameter specifies which pairs of bonded atoms should
be excluded from non-bonded
interactions.  With the value of {\tt none}, no bonded pairs of atoms 
will be excluded.  With the value of {\tt 1-2}, all atom pairs that
are directly connected via a linear bond will be excluded.  With the
value of {\tt 1-3}, all {\tt 1-2} pairs will be excluded along with
all pairs of atoms that are bonded to a common
third atom (i.e., if atom A is bonded to atom B and atom B is bonded
to atom C, then the atom pair A-C would be excluded).
With the value of {\tt 1-4}, all {\tt 1-3} pairs will be excluded along
with all pairs connected by a set of two bonds (i.e., if atom A is bonded
to atom B, and atom B is bonded to atom C, and atom C is bonded to
atom D, then the atom pair A-D would be excluded).  With the value
of {\tt scaled1-4}, all {\tt 1-3} pairs are excluded and all pairs
that match the {\tt 1-4} criteria are modified.  The electrostatic
interactions for such pairs are modified by the constant factor
defined by {\tt 1-4scaling}.  
The van der Waals interactions are modified
by using the special 1-4 parameters defined in the parameter files.
The value of {\tt scaled1-4} is necessary to enable the modified
1-4 VDW parameters present in the CHARMM parameter files.
}

\item
\NAMDCONFWDEF{1-4scaling}{scaling factor for 1-4 electrostatic interactions}
{0 $\leq$ decimal $\leq$ 1}{1.0}
{Scaling factor for 1-4 electrostatic interactions.
This factor is only used when the
{\tt exclude} parameter is set to {\tt scaled1-4}.  In this case, this
factor is used to modify the electrostatic interactions between 1-4 atom
pairs.  If the {\tt exclude} parameter is set to anything but 
{\tt scaled1-4}, this parameter has no effect regardless of its value.}

\item
\NAMDCONFWDEF{dielectric}{dielectric constant for system}
{decimal $\geq$ 1.0}{1.0}
{Dielectric constant for the system.  A value of 1.0 implies no modification
of the electrostatic interactions.  Any larger value will lessen the
electrostatic forces acting in the system.}

\item
\NAMDCONFWDEF{nonbondedScaling}{scaling factor for nonbonded forces}
{decimal $\geq$ 0.0}{1.0}
{Scaling factor for electrostatic and van der Waals forces.
A value of 1.0 implies no modification of the interactions.
Any smaller value will lessen the
nonbonded forces acting in the system.}

\item
\NAMDCONFWDEF{vdwGeometricSigma}{use geometric mean to combine L-J sigmas}
{{\tt yes} or {\tt no}}{{\tt no}}
{Use geometric mean, as required by \index{OPLS} OPLS, rather than
traditional arithmetic mean when combining Lennard-Jones sigma parameters
for different atom types.}

\item
\NAMDCONFWDEF{limitdist}
{maximum distance between pairs for limiting interaction strength(\AA)}
{non-negative decimal}
{{\tt 0.}}
{
The electrostatic and van der Waals potential functions diverge
as the distance between two atoms approaches zero.
The potential for atoms closer than {\tt limitdist} is instead
treated as $a r^2 + c$ with parameters chosen to match the
force and potential at {\tt limitdist}.
This option should primarily be useful for alchemical free energy
perturbation calculations, since it makes the process of creating
and destroying atoms far less drastic energetically.
The larger the value of {\tt limitdist} the more the maximum force
between atoms will be reduced.
In order to not alter the other interactions in the simulation,
{\tt limitdist} should be less than the closest approach
of any non-bonded pair of atoms; 1.3\,\AA\ appears to satisfy this
for typical simulations but the user is encouraged to experiment.
There should be no performance impact from enabling this feature.
}

\item
\NAMDCONFWDEF{LJcorrection}
{Apply long-range corrections to the system energy and virial to
account for neglected vdW forces?}{{\tt yes} or {\tt no}}{{\tt no}}
{Apply an analytical correction to the reported vdW energy and virial
that is equal to the amount lost due to switching and cutoff of the LJ
potential. The correction will use the average of vdW parameters for
all particles in the system and assume a constant, homogeneous
distribution of particles beyond the switching distance. See 
\cite{Shirts2007} for details (the equations used in the NAMD
implementation are slightly different due to the use of a different
switching function). Periodic boundary conditions are required to make
use of tail corrections.
{\bf LJcorrection as implemented is inconsistent with vdwForceSwitching.}}

\end{itemize}


\subsubsection{PME parameters}

PME stands for Particle Mesh Ewald and is an efficient
full electrostatics method for use with periodic boundary conditions.
None of the parameters should affect energy conservation, although they may affect the accuracy of the results and momentum conservation.

\begin{itemize}

\item
\NAMDCONFWDEF{PME}{Use particle mesh Ewald for electrostatics?}{{\tt yes} or {\tt no}}{{\tt no}}
{Turns on particle mesh Ewald.}

\item
\NAMDCONFWDEF{PMETolerance}{PME direct space tolerance}{positive decimal}{$10^{-6}$}
{Affects the value of the Ewald coefficient and the overall accuracy of the results.}

\item
\NAMDCONFWDEF{PMEInterpOrder}{PME interpolation order}{positive integer}{4 (cubic)}
{Charges are interpolated onto the grid and forces are interpolated off using this many points, equal to the order of the interpolation function plus one.}

\item
\NAMDCONF{PMEGridSpacing}{maximum space between grid points}{positive real}
{The grid spacing partially determines the accuracy and efficiency of PME.
If any of the grid sizes below are not set, then PMEGridSpacing must be set
(recommended value is 1.0 \AA ) and will be used to calculate them.
If a grid size is set, then the grid spacing must be
at least PMEGridSpacing (if set, or a very large default of 1.5).}

\item
\NAMDCONF{PMEGridSizeX}{number of grid points in x dimension}{positive integer}
{The grid size partially determines the accuracy and efficiency of PME.
For speed, {\tt PMEGridSizeX} should have only small integer factors (2, 3 and 5).}

\item
\NAMDCONF{PMEGridSizeY}{number of grid points in y dimension}{positive integer}
{The grid size partially determines the accuracy and efficiency of PME.
For speed, {\tt PMEGridSizeY} should have only small integer factors (2, 3 and 5).}

\item
\NAMDCONF{PMEGridSizeZ}{number of grid points in z dimension}{positive integer}
{The grid size partially determines the accuracy and efficiency of PME.
For speed, {\tt PMEGridSizeZ} should have only small integer factors (2, 3 and 5).}

\item
\NAMDCONFWDEF{PMEProcessors}{processors for FFT and reciprocal sum}{positive integer}{larger of x and y grid sizes up to all available processors}
{For best performance on some systems and machines, it may be necessary to
restrict the amount of parallelism used.  Experiment with this parameter if
your parallel performance is poor when PME is used.}

\item
\NAMDCONFWDEF{FFTWEstimate}{Use estimates to optimize FFT?}{{\tt yes} or {\tt no}}{{\tt no}}
{Do not optimize FFT based on measurements, but on FFTW rules of thumb.
This reduces startup time, but may affect performance.}

\item
\NAMDCONFWDEF{FFTWUseWisdom}{Use FFTW wisdom archive file?}{{\tt yes} or {\tt no}}{{\tt yes}}
{Try to reduce startup time when possible by reading FFTW ``wisdom'' from a file, and saving wisdom generated by performance measurements to the same file for future use.
This will reduce startup time when running the same size PME grid on the same number of processors as a previous run using the same file.}

\item
\NAMDCONFWDEF{FFTWWisdomFile}{name of file for FFTW wisdom archive}{file name}{FFTW\_NAMD\_{\em version}\_{\em platform}.txt}
{File where FFTW wisdom is read and saved.
If you only run on one platform this may be useful to reduce startup times for all runs.
The default is likely sufficient, as it is version and platform specific.}

\end{itemize}

\subsubsection{Full direct parameters}

The direct computation of electrostatics 
is not intended to be used during 
real calculations, but rather as a testing or 
comparison measure.  Because of the ${\mathcal O}(N^2)$ 
computational complexity for performing 
direct calculations, this is {\it much} 
slower than using DPMTA or PME to compute full 
electrostatics for large systems.
In the case of periodic boundary conditions,
the nearest image convention is used rather than a
full Ewald sum.

\begin{itemize}

\item
\NAMDCONFWDEF{FullDirect}{calculate full electrostatics directly?}{{\tt yes} or {\tt no}}{{\tt no}}
{Specifies whether or not direct computation of 
full electrostatics should be performed.}

\end{itemize}


\subsubsection{DPME parameters}

{\em DPME is an implementation of PME that is no longer
included in the released NAMD binaries.
We recommend that you use the current PME implementation.}
\begin{itemize}
\item
\NAMDCONFWDEF{useDPME}{Use old DPME code?}{{\tt yes} or {\tt no}}{{\tt no}}
{Switches to old DPME implementation of particle mesh Ewald.
The new code is faster and allows non-orthogonal cells so you
probably just want to leave this option turned off.  If you set
{\tt cellOrigin} to something other than $(0,0,0)$ the energy may differ
slightly between the old and new implementations.
{\em DPME is no longer included in released binaries.}
}
\end{itemize}


\subsubsection{DPMTA parameters}

{\em DPMTA is no longer included in the released NAMD binaries.
We recommend that you instead use PME with a periodic system because
it conserves energy better, is more efficient, and is better parallelized.
If you must have the fast multipole algorithm you may compile \NAMD\ yourself.} 

These parameters control the options to DPMTA, an algorithm
used to provide full electrostatic interactions.  DPMTA is a
modified version of the FMA (Fast Multipole Algorithm) and, 
unfortunately, most of the parameters still refer to FMA
rather than DPMTA for historical reasons.  Don't be confused!
\prettypar
For a further description of how exactly full electrostatics
are incorporated into \NAMD, see Section \ref{section:mts}.
For a greater level of detail about DPMTA and the specific
meaning of its options, see the DPMTA distribution which is
available via anonymous FTP from the site {\tt ftp.ee.duke.edu}
in the directory {\tt /pub/SciComp/src}.

\begin{itemize}

\item
\NAMDCONFWDEF{FMA}{use full electrostatics?}{{\tt on} or {\tt off}}{{\tt off}}
{Specifies whether or not 
the DPMTA algorithm from Duke University should be used 
to compute the full electrostatic interactions.  If set to 
{\tt on}, DPMTA will be used with a multiple timestep integration scheme 
to provide full electrostatic interactions as detailed in Section 
\ref{section:mts}.  {\em DPMTA is no longer included in released binaries.}}

\item
\NAMDCONFWDEF{FMALevels}{number of levels to use in multipole expansion}{positive integer}{5}
{Number of levels to use for the multipole expansion.  This parameter
is only used if {\tt FMA} is set to {\tt on}.  
A value of 4 should be sufficient for systems with less than 10,000 atoms.  
A value of 5 or greater should be used for larger systems. }

\item
\NAMDCONFWDEF{FMAMp}{number of multipole terms to use for FMA}{positive integer}{8}
{Number of terms to use in the multipole expansion.  
This parameter is only used if {\tt FMA} is set to {\tt on}.  
If the {\tt FMAFFT} is set to {\tt on}, then this value must 
be a multiple of 4.  The default value of 8 should be suitable
for most applications.}

\item
\NAMDCONFWDEF{FMAFFT}{use DPMTA FFT enhancement?}{{\tt on} or {\tt off}}{{\tt on}}
{Specifies whether or not the DPMTA code should use the FFT enhancement 
feature.  This parameter is only used if {\tt FMA} is set to {\tt on}.  
If {\tt FMAFFT} is set to {\tt on}, the value of {\tt FMAMp} must be 
set to a multiple of 4.  
This feature offers substantial benefits only for values 
of {\tt FMAMp} of 8 or greater.  This feature will substantially 
increase the amount of memory used by DPMTA.}

%%  REMOVE THIS AS A DOCUMENTED FEATURE.  WE USE stepspercycle INSTEAD.  
%%
%%  \item
%%  \NAMDCONF{FMAfrequency}{number of timesteps between DPMTA calculations}{positive integer}
%%  {This parameter specifies the number of timesteps between each
%%  invocation of the DPMTA algorithm.}

\item
\NAMDCONFWDEF{FMAtheta}{DPMTA theta parameter (radians)}{decimal}{0.715}
{This parameter specifies the value of the theta parameter
used in the DPMTA calculation.  The default value is based on
recommendations by the developers of the code.}

\item
\NAMDCONFWDEF{FMAFFTBlock}{blocking factor for FMA FFT}{positive integer}{4}
{The blocking factor for the FFT enhancement to DPMTA.
This parameter is only used if both {\tt FMA} and {\tt FMAFFT} 
are set to {\tt on}.  The default value of 4 should be suitable
for most applications.}

\end{itemize}



\subsection{Water Models}
\label{section:water_models}

\subsubsection{Water model parameters}

\NAMD~currently supports the 3-site TIP3P water model and the 4-site TIP4P water model.  TIP3P is the current default water model.  Usage of alternate water models is described below. 

\begin{itemize}

  \item
    \NAMDCONFWDEF{waterModel}{using which water model?}{ {\tt tip3} or {\tt tip4}}{ {\tt tip3}}
    {Specifies if the TIP3P or TIP4P water model is to be used.  When using the TIP4P water model, the ordering of atoms within each TIP4P water molecule must be oxygen, hydrogen, hydrogen, lone pair.  Alternate orderings will fail. } 

\end{itemize}


\subsection{Constraints and Restraints}
%\label{section:config_add}

\subsubsection{Bond constraint parameters}
\label{section:rigidBonds}
\begin{itemize}
\item
\NAMDCONFWDEF{rigidBonds}{controls if and how ShakeH is used}{{\tt none},
{\tt water}, {\tt all}}{{\tt none}} 
{When {\tt water} is selected, the hydrogen-oxygen and hydrogen-hydrogen
distances in waters are constrained to the nominal length or angle given
in the parameter file, making the molecules completely rigid.
When {\tt rigidBonds} is {\tt all}, waters are made rigid as described above
and the bond between each hydrogen and the (one) atom to which
it is bonded is similarly constrained.
For the default case {\tt none}, no lengths are constrained.
}

\item
\NAMDCONFWDEF{rigidTolerance}{allowable bond-length error for ShakeH (\AA)}
{positive decimal}{1.0e-8}
{
The ShakeH algorithm is assumed to have converged when all constrained
bonds differ from the nominal bond length by less than this amount.
}

\item
\NAMDCONFWDEF{rigidIterations}{maximum ShakeH iterations}{positive integer}{100}
{
The maximum number of iterations ShakeH will perform before giving up
on constraining the bond lengths.  If the bond lengths do not
converge, a warning message is printed, and the atoms are left at the
final value achieved by ShakeH.  
Although the default value is 100, 
convergence is usually reached after fewer than 10 iterations.
}

\item
\NAMDCONFWDEF{rigidDieOnError}{maximum ShakeH iterations}{{\tt on} or {\tt off}}
{{\tt on}}
{
Exit and report an error if rigidTolerance is not achieved after rigidIterations.
}

\item
\NAMDCONFWDEF{useSettle}{Use SETTLE for waters.}{{\tt on} or {\tt off}}
{{\tt on}}
{
If rigidBonds are enabled then use the non-iterative SETTLE algorithm to
keep waters rigid rather than the slower SHAKE algorithm.
}
\end{itemize}

\subsubsection{Harmonic restraint parameters}

The following describes the parameters for the 
harmonic restraints feature of \NAMD.
%Actually, this feature 
%should be referred to as harmonic restraints rather than 
%constraints, but
For historical reasons the terminology of 
``harmonic constraints'' has been carried over from X-PLOR.  
This feature allows a harmonic restraining force to be applied 
to any set of atoms in the simulation.

\begin{itemize}

\item
\NAMDCONFWDEF{constraints}{are constraints active?}{{\tt on} or {\tt off}}{{\tt off}}
{Specifies whether or not harmonic constraints are active.  If it 
is set to {\tt off}, then no harmonic constraints are computed.  
If it is set to {\tt on}, then 
harmonic constraints are calculated using the values specified 
by the parameters {\tt consref}, {\tt conskfile}, {\tt conskcol}, 
and {\tt consexp}.}

\item
\NAMDCONFWDEF{consexp}{exponent for harmonic constraint energy function}{positive, even integer}{2}
{Exponent to be use in the harmonic constraint energy function.  
This value must be a positive integer, and only even values really make 
sense.  This parameter is used only if {\tt constraints} is set to 
{\tt on}.}

\item
\NAMDCONF{consref}{PDB file containing constraint reference positions}{UNIX file name}
{PDB file to use for reference positions for harmonic constraints.  
Each atom that has an active constraint will be constrained about 
the position specified in this file.}

\item
\NAMDCONF{conskfile}{PDB file containing force constant values}{UNIX filename}
{PDB file to use for force constants for 
harmonic constraints.}

\item
\NAMDCONF{conskcol}{column of PDB file containing force constant}{{\tt X}, {\tt Y}, {\tt Z}, {\tt O}, or {\tt B}}
{Column of the PDB file to use for the harmonic constraint force constant.
This parameter may specify any of the floating point fields of the PDB file, 
either X, Y, Z, occupancy, or beta-coupling (temperature-coupling).  
Regardless of which column is used, a value of 0 indicates that the atom 
should not be constrained.  
Otherwise, the value specified is used as the force constant for 
that atom's restraining potential.}

\item
\NAMDCONFWDEF{constraintScaling}{scaling factor for harmonic constraint energy function}{positive}{1.0}
{The harmonic constraint energy function is multiplied by this parameter,
making it possible to gradually turn off constraints during equilibration.
This parameter is used only if {\tt constraints} is set to 
{\tt on}.}

\item
\NAMDCONFWDEF{selectConstraints}{Restrain only selected Cartesian components of the coordinates?}{{\tt on} or {\tt off}}{{\tt off}}
{This option is useful to restrain the positions of atoms to a plane or a line in space. If active,
 this option will ensure that only selected Cartesian components of the coordinates are restrained.
 E.g.: Restraining the positions of atoms to their current z values with no restraints
 in x and y will allow the atoms to move in the x-y plane while retaining their original z-coordinate.
 Restraining the x and y values will lead to free motion only along the z coordinate.}

\item
\NAMDCONFWDEF{selectConstrX}{Restrain X components of coordinates}{{\tt on} or {\tt off}}{{\tt off}}
{Restrain the Cartesian x components of the positions.}
\item
\NAMDCONFWDEF{selectConstrY}{Restrain Y components of coordinates}{{\tt on} or {\tt off}}{{\tt off}}
{Restrain the Cartesian y components of the positions.}
\item
\NAMDCONFWDEF{selectConstrZ}{Restrain Z components of coordinates}{{\tt on} or {\tt off}}{{\tt off}}
{Restrain the Cartesian z components of the positions.}

\end{itemize}

\subsubsection{Fixed atoms parameters}

Atoms may be held fixed during a simulation.  \NAMD\ avoids calculating most interactions in which all affected atoms are fixed unless {\tt fixedAtomsForces} is specified.

\begin{itemize}

\item
\NAMDCONFWDEF{fixedAtoms}{are there fixed atoms?}{{\tt on} or {\tt off}}{{\tt off}}
{Specifies whether or not fixed atoms are present.} 

\item
\NAMDCONFWDEF{fixedAtomsForces}{are forces between fixed atoms calculated?}{{\tt on} or {\tt off}}{{\tt off}}
{Specifies whether or not forces between fixed atoms are calculated.  This option is required to turn fixed atoms off in the middle of a simulation.
These forces will affect the pressure calculation, and you should leave this option off when using constant pressure if the coordinates of the fixed atoms have not been minimized.
The use of constant pressure with significant numbers of fixed atoms is not recommended.}

\item
\NAMDCONFWDEF{fixedAtomsFile}{PDB file containing fixed atom parameters}
{UNIX filename}{{\tt coordinates}}
{PDB file to use for the fixed atom flags for each atom.  
If this parameter is not specified, then 
the PDB file specified by {\tt coordinates} is used.}

\item
\NAMDCONFWDEF{fixedAtomsCol}{column of PDB containing fixed atom parameters}
{{\tt X}, {\tt Y}, {\tt Z}, {\tt O}, or {\tt B}}{{\tt O}} 
{Column of the PDB file to use for the containing fixed atom parameters for 
each atom.  The coefficients can be read from any 
floating point column of the PDB file.  
A value of 0 indicates that the atom is not fixed.}

\end{itemize}


