\section{Additional Simulation Parameters}
\label{section:add}


\subsection{Constraints and Restraints}
\label{section:config_add}

\subsubsection{Harmonic constraint parameters}

The following describes the parameters for the 
harmonic constraints feature of \NAMD.  Actually, this feature 
should be referred to as harmonic restraints rather than 
constraints, but for historical reasons the terminology of 
harmonic constraints has been carried over from X-PLOR.  
This feature allows a harmonic restraining force to be applied 
to any set of atoms in the simulation.  For further details, 
see the \PG.  

\begin{itemize}

\item
\NAMDCONFWDEF{constraints}{are constraints active?}{\verb!on! or \verb!off!}{\verb!off!}
{Specifies whether or not harmonic constraints are active.  If it 
is set to \verb!off!, then no harmonic constraints are computed.  
If it is set to \verb!on!, then 
harmonic constraints are calculated using the values specified 
by the parameters \verb!consref!, \verb!conskfile!, \verb!conskcol!, 
and \verb!consexp!.}

\item
\NAMDCONFWDEF{consexp}{exponent for harmonic constraint energy function}{positive, even integer}{2}
{Exponent to be use in the harmonic constraint energy function.  
This value must be a positive integer, and only even values really make 
sense.  This parameter is used only if \verb!constraints! is set to 
\verb!on!.}

\item
\NAMDCONFWDEF{consref}{PDB file containing constraint reference positions}{UNIX file name}{\verb!coordinates!}
{PDB file to use for reference positions for harmonic constraints.  
Each atom that has an active constraint will be constrained about 
the position specified in this file.  If no value is given and constraints 
are active, then the same PDB file specified by \verb!coordinates! will be 
used instead, constraining atoms about their initial positions.}

\item
\NAMDCONFWDEF{conskfile}{PDB file containing force constant values}{UNIX filename} {\verb!coordinates!}
{PDB file to use for force constants for 
harmonic constraints.  
If this parameter is not specified, then 
the PDB file containing initial coordinates specified by 
\verb!coordinates! is used.}

\item
\NAMDCONFWDEF{conskcol}{column of PDB file containing force constant}{\verb!X!, \verb!Y!, \verb!Z!, \verb!O!, or \verb!B!}{\verb!O!}
{Column of the PDB file to use for the harmonic constraint force constant.
This parameter may specify any of the floating point fields of the PDB file, 
either X, Y, Z, occupancy, or beta-coupling (temperature-coupling).  
Regardless of which column is used, a value of 0 indicates that the atom 
should not be constrained.  
Otherwise, the value specified is used as the force constant for 
that atom's restraining potential.}

\item
\NAMDCONFWDEF{selectConstraints}{Restrain only selected Cartesian components of the coordinates?}{\verb!on! or \verb!off!}{\verb!off!}
{This option is useful to restrain the positions of atoms to a plane or a line in space. If active,
 this option will ensure that only selected Cartesian components of the coordinates are restrained.
 E.g.: Restraining the positions of atoms to their current z values with no restraints
 in x and y will allow the atoms to move in the x-y plane while retaining their original z-coordinate.
 Restraining the x and y values will lead to free motion only along the z coordinate.}

\item
\NAMDCONFWDEF{selectConstrX}{Restrain X components of coordinates}{\verb!on! or \verb!off!}{\verb!off!}
{Restrain the Cartesian x components of the positions.}
\item
\NAMDCONFWDEF{selectConstrY}{Restrain Y components of coordinates}{\verb!on! or \verb!off!}{\verb!off!}
{Restrain the Cartesian y components of the positions.}
\item
\NAMDCONFWDEF{selectConstrZ}{Restrain Z components of coordinates}{\verb!on! or \verb!off!}{\verb!off!}
{Restrain the Cartesian z components of the positions.}

\end{itemize}

\subsubsection{Fixed atoms parameters}

Atoms may be held fixed during a simulation.  \NAMD\ avoids calculating most interactions in which all affected atoms are fixed.

\begin{itemize}

\item
\NAMDCONFWDEF{fixedAtoms}{are there fixed atoms?}{\verb!on! or \verb!off!}{\verb!off!}
{Specifies whether or not fixed atoms are present.} 

\item
\NAMDCONFWDEF{fixedAtomsFile}{PDB file containing fixed atom parameters}%
{UNIX filename}{\verb!coordinates!}
{PDB file to use for the fixed atom flags for each atom.  
If this parameter is not specified, then 
the PDB file specified by \verb!coordinates! is used.}

\item
\NAMDCONFWDEF{fixedAtomsCol}{column of PDB containing fixed atom parameters}
{\verb!X!, \verb!Y!, \verb!Z!, \verb!O!, or \verb!B!}{\verb!O!} 
{Column of the PDB file to use for the containing fixed atom parameters for 
each atom.  The coefficients can be read from any 
floating point column of the PDB file.  
A value of 0 indicates that the atom is not fixed.}

\end{itemize}

\subsection{Energy Minimization}

\subsubsection{Velocity quenching parameters}

As described in the \PG, \NAMD\ has the capability
of performing energy minimization using a simple quenching
scheme.   While this algorithm is not the most rapidly convergent, it
is sufficient for most applications.  There are only two parameters
for minimization:  one to activate minimization and another
to specify the maximum movement of any atom.  

\begin{itemize}

\item
\NAMDCONFWDEF{minimization}{Perform energy minimization?}{\verb!on! or \verb!off!}{\verb!off!}
{Turns energy minimization \verb!on! or \verb!off!.}

\item
\NAMDCONFWDEF{maximumMove}{maximum distance an atom can move during each step (\AA)}%
{positive decimal}%
{$0.75\times\mbox{\verb!cutoff!}/\mbox{\verb!stepsPerCycle!}$}
{Maximum distance that an atom can move during any single timestep of
minimization.  This is to insure that atoms do not go flying off into
space during the first few timesteps when the largest energy conflicts
are resolved.}

\end{itemize}

\subsection{Temperature Control and Equilibration}

\subsubsection{Langevin dynamics parameters}

As described in the \PG, \NAMD\ is capable
of performing Langevin dynamics, where additional damping and
random forces are introduced to the system.  This capability
is based on that implemented in X-PLOR which is detailed
in the X-PLOR {\it User's Manual} \mycite{(Br\"unger, 1992)}{BRUN92b},
although a different integrator is used.

\begin{itemize}

\item
\NAMDCONFWDEF{langevin}{use Langevin dynamics?}{\verb!on! or \verb!off!}{\verb!off!}
{Specifies whether or not Langevin dynamics active.  
If set to \verb!on!, then the parameter \verb!langevinTemp! must be set 
and the parameters \verb!langevinFile! and \verb!langevinCol! can
optionally be set to control the behavior of this feature.} 

\item
\NAMDCONF{langevinTemp}{temperature for Langevin calculations (K)}{positive decimal}
{Temperature to which atoms affected by Langevin dynamics will be adjusted.  
This temperature will be roughly maintained across the affected atoms 
through the addition of friction and random forces.}

\item
\NAMDCONFWDEF{langevinFile}{PDB file containing Langevin parameters}%
{UNIX filename}{\verb!coordinates!}
{PDB file to use for the Langevin coupling coefficients for each atom.  
If this parameter is not specified, then 
the PDB file specified by \verb!coordinates! is used.}

\item
\NAMDCONFWDEF{langevinCol}{column of PDB from which to read coefficients}
{\verb!X!, \verb!Y!, \verb!Z!, \verb!O!, or \verb!B!}{\verb!O!} 
{Column of the PDB file to use for the Langevin coupling coefficients for 
each atom.  The coefficients can be read from any 
floating point column of the PDB file.  
A value of 0 indicates that the atom will remain unaffected.}

\end{itemize}

\subsubsection{Temperature coupling parameters}

As described in the \PG, \NAMD\ is capable
of performing temperature coupling, in which forces are added or 
reduced to simulate the coupling of the system to a heat bath 
of a specified temperature.  
This capability is based on that implemented in X-PLOR which is detailed
in the X-PLOR {\it User's Manual} \mycite{(Br\"unger, 1992)}{BRUN92b}.

\begin{itemize}

\item
\NAMDCONFWDEF{tCouple}{perform temperature coupling?}{\verb!on! or \verb!off!}{\verb!off!}
{Specifies whether or not temperature coupling is active.  
If set to \verb!on!, then the parameter \verb!tCoupleTemp! must be set and 
the parameters \verb!tCoupleFile! and \verb!tCoupleCol! can 
optionally be set to control the behavior of this feature.} 

\item
\NAMDCONF{tCoupleTemp}{temperature for heat bath (K)}{positive decimal}
{Temperature to which atoms affected 
by temperature coupling will be adjusted.  
This temperature will be roughly maintained across the affected atoms 
through the addition of forces.}

\item
\NAMDCONFWDEF{tCoupleFile}{PDB file with tCouple parameters}%
{UNIX filename}{\verb!coordinates!}%
{PDB file to use for the temperature coupling coefficient for each atom.  
If this parameter is not specified, then 
the PDB file specified by \verb!coordinates! is used.} 

\item
\NAMDCONFWDEF{tCoupleCol}{column of PDB from which to read coefficients}
{\verb!X!, \verb!Y!, \verb!Z!, \verb!O!, or \verb!B!}{\verb!O!} 
{Column of the PDB file to use for the temperature coupling coefficient for 
each atom.  This value can be read from any 
floating point column of the PDB file.  
A value of $0$ indicates that the atom will remain unaffected.}

\end{itemize}

\subsubsection{Temperature rescaling parameters}

\NAMD\ allows equilibration of a system by means of temperature 
rescaling.  Using this method, all of the velocities in the system 
are periodically rescaled so that the entire system is set to the 
desired temperature.  The following parameters specify how often 
and to what temperature this rescaling is performed.  

\begin{itemize}

\item
\NAMDCONF{rescaleFreq}{number of timesteps between temperature rescaling}{positive integer}
{The equilibration feature of \NAMD\ is activated by 
specifying the number of timesteps between each temperature rescaling.  
If this value is given, then the \verb!rescaleTemp! parameter must also 
be given to specify the target temperature. }

\item
\NAMDCONF{rescaleTemp}{temperature for equilibration (K)}{positive decimal}
{The temperature to which all velocities will be rescaled
every \verb!rescaleFreq! timesteps.  
This parameter is valid only if \verb!rescaleFreq! has been set.}

\end{itemize}

\subsubsection{Temperature reassignment parameters}

\NAMD\ allows equilibration of a system by means of temperature 
reassignment.  Using this method, all of the velocities in the system 
are periodically reassigned so that the entire system is set to the 
desired temperature.  The following parameters specify how often 
and to what temperature this reassignment is performed.  

\begin{itemize}

\item
\NAMDCONF{reassignFreq}{number of timesteps between temperature reassignment}{positive integer}
{The equilibration feature of \NAMD\ is activated by 
specifying the number of timesteps between each temperature reassignment.  
If this value is given, then the \verb!reassignTemp! parameter must also 
be given to specify the target temperature. }

\item
\NAMDCONF{reassignTemp}{temperature for equilibration (K)}{positive decimal}
{The temperature to which all velocities will be reassigned
every \verb!reassignFreq! timesteps.  
This parameter is valid only if \verb!reassignFreq! has been set.}

\item
\NAMDCONFWDEF{reassignIncr}{temperature increment for equilibration (K)}{positive decimal}{0}
{In order to allow simulated annealing or other slow heating/cooling protocols, \verb!reassignIncr! will be added to \verb!reassignTemp! after each reassignment.
This parameter is valid only if \verb!reassignFreq! has been set.}

\end{itemize}

\subsection{Boundary Conditions}

\subsubsection{Spherical harmonic boundary conditions}

\NAMD\ provides spherical harmonic boundary conditions.  These 
boundary conditions can consist of a single potential or a 
combination of two potentials as described in the \PG.  
The following parameters are used to define these boundary conditions.  

\begin{itemize}

\item
\NAMDCONFWDEF{sphericalBC}{use spherical boundary conditions?}{\verb!on! or \verb!off!}{\verb!off!}
{Specifies whether or not spherical boundary conditions 
are to be applied to the system.  If 
set to \verb!on!, then \verb!sphericalBCr1! and \verb!sphericalBCk1! 
must be defined, and \verb!sphericalBCexp1!, \verb!sphericalBCr2!, 
\verb!sphericalBCk2!, and \verb!sphericalBCexp2! can optionally be 
defined.}

\item
\NAMDCONF{sphericalBCr1}{radius for first boundary condition (\AA)}{positive decimal}
{Distance at which the first potential of the boundary conditions takes
effect.  This distance is a radius from the center of mass.}

\item
\NAMDCONF{sphericalBCk1}{force constant for first potential}{non-zero decimal}
{Force constant for the first harmonic potential.  A positive
value will push atoms toward the center of mass, and a negative
value will pull atoms away from the center of mass.}

\item
\NAMDCONFWDEF{sphericalBCexp1}{exponent for first potential}{positive, even integer}{2}
{Exponent for first boundary potential.  The only likely values to
use are 2 and 4.}

\item
\NAMDCONF{sphericalBCr2}{radius for second boundary condition (\AA)}{positive decimal}
{Distance at which the second potential of the boundary conditions takes
effect.  This distance is a radius from the center of mass.
If this parameter is defined, then \verb!spericalBCk2! must also
be defined.}

\item
\NAMDCONF{sphericalBCk2}{force constant for second potential}{non-zero decimal}
{Force constant for the second harmonic potential.  A positive
value will push atoms toward the center of mass, and a negative
value will pull atoms away from the center of mass.}

\item
\NAMDCONFWDEF{sphericalBCexp2}{exponent for second potential}{positive, even integer}{2}
{Exponent for second boundary potential.  The only likely values to
use are 2 and 4.}

\end{itemize}

\subsubsection{Cylindrical harmonic boundary conditions}

\NAMD\ provides cylindrical harmonic boundary conditions.  These 
boundary conditions can consist of a single potential or a 
combination of two potentials as described in the \PG.  
The following parameters are used to define these boundary conditions.  

\begin{itemize}

\item
\NAMDCONFWDEF{cylindricalBC}{use cylindrical boundary conditions?}{\verb!on! or \verb!off!}{\verb!off!}
{Specifies whether or not cylindrical boundary conditions 
are to be applied to the system.  If 
set to \verb!on!, then \verb!cylindricalBCr1!, \verb!cylindricalBCl1! and \verb!cylindricalBCk1! 
must be defined, and \verb!cylindricalBCAxis!, \verb!cylindricalBCCenter!, \verb!cylindricalBCexp1!, \verb!cylindricalBCr2!, \verb!cylindricalBCl2!,
\verb!cylindricalBCk2!, and \verb!cylindricalBCexp2! can optionally be 
defined.}

\item
\NAMDCONF{cylindricalBCCenter}{center of  cylinder (\AA)}{position}
{Location around which cylinder is centered.  If not specified, center of mass is used.}

\item
\NAMDCONF{cylindricalBCAxis}{axis of  cylinder (\AA)}{\verb!x!, \verb!y!, or \verb!z!}
{Axis along which cylinder is aligned.}

\item
\NAMDCONF{cylindricalBCr1}{radius for first boundary condition (\AA)}{positive decimal}
{Distance at which the first potential of the boundary conditions takes
effect along the non-axis plane of the cylinder.}

\item
\NAMDCONF{cylindricalBCl1}{distance along cylinder axis for first boundary condition (\AA)}{positive decimal}
{Distance at which the first potential of the boundary conditions takes
effect along the cylinder axis.}

\item
\NAMDCONF{cylindricalBCk1}{force constant for first potential}{non-zero decimal}
{Force constant for the first harmonic potential.  A positive
value will push atoms toward the center, and a negative
value will pull atoms away from the center.}

\item
\NAMDCONFWDEF{cylindricalBCexp1}{exponent for first potential}{positive, even integer}{2}
{Exponent for first boundary potential.  The only likely values to
use are 2 and 4.}

\item
\NAMDCONF{cylindricalBCr2}{radius for second boundary condition (\AA)}{positive decimal}
{Distance at which the second potential of the boundary conditions takes
effect along the non-axis plane of the cylinder.
If this parameter is defined, then \verb!cylindricalBCl2! and \verb!spericalBCk2! must also
be defined.}

\item
\NAMDCONF{cylindricalBCl2}{radius for second boundary condition (\AA)}{positive decimal}
{Distance at which the second potential of the boundary conditions takes
effect along the cylinder axis.
If this parameter is defined, then \verb!cylindricalBCr2! and \verb!spericalBCk2! must also
be defined.}

\item
\NAMDCONF{cylindricalBCk2}{force constant for second potential}{non-zero decimal}
{Force constant for the second harmonic potential.  A positive
value will push atoms toward the center, and a negative
value will pull atoms away from the center.}

\item
\NAMDCONFWDEF{cylindricalBCexp2}{exponent for second potential}{positive, even integer}{2}
{Exponent for second boundary potential.  The only likely values to
use are 2 and 4.}

\end{itemize}

%\paragraph{Applied electric field}
%
%\NAMD\ provides the ability to apply a constant electric field to the molecular
%system being simulated.  There are two parameters that control this feature.
%
%\begin{itemize}
%
%\item
%\NAMDCONFWDEF{eFieldOn}{apply electric field?}{\verb!yes! or %\verb!no!}{\verb!no!}
%{Specifies whether or not an electric field is applied.}
%
%\item
%\NAMDCONF{eField}{electric field vector}{vector of decimals}
%{Vector which describes the electric field to be applied.}
%
%\end{itemize}

\subsubsection{Periodic boundary conditions}

\NAMD\ provides periodic boundary conditions in 1, 2 or 3 dimensions.
The basis vectors of the periodic cell must be along the x, y, and z axes.  
This restriction will be eliminated in future versions of \NAMD.
The following parameters are used to define these boundary conditions.  

\begin{itemize}

\item
\NAMDCONFWDEF{cellBasisVector1}{basis vector for periodic boundaries (\AA)}{vector along x axis}{0 0 0}
{Specifies whether or not periodic boundary conditions 
are to be applied to the system in the x direction.}

\item
\NAMDCONFWDEF{cellBasisVector2}{basis vector for periodic boundaries (\AA)}{vector along y axis}{0 0 0}
{Specifies whether or not periodic boundary conditions 
are to be applied to the system in the y direction.}

\item
\NAMDCONFWDEF{cellBasisVector3}{basis vector for periodic boundaries (\AA)}{vector along z axis}{0 0 0}
{Specifies whether or not periodic boundary conditions 
are to be applied to the system in the z direction.}

\item
\NAMDCONFWDEF{cellOrigin}{center of periodic cell (\AA)}{position}{0 0 0}
{When position rescaling is used to control pressure, this location will remain constant.  Also used as the center of the cell for wrapped output coordinates.}

\item
\NAMDCONF{extendedSystem}{XSC file to read cell parameters from}{file name}
{In addition to .coor and .vel output files, \NAMD\ generates a .xsc (eXtended System Configuration) file which contains the periodic cell parameters and extended system variables, such as the strain rate in constant pressure simulations.  Periodic cell parameters will be read from this file if this option is present, ignoring the above parameters.}

\item
\NAMDCONF{XSTfile}{XST file to read cell trajectory to}{file name}
{\NAMD\ can also generate a .xst (eXtended System Trajectory) file which contains a record of the periodic cell parameters and extended system variables during the simulation.}

\item
\NAMDCONFWDEF{XSTFreq}{how often to append state to XST file}{positive integer}{1}
{Like the \verb!DCDfreq! option, controls how often the extended system configuration will be appended to the XST file.}

\item
\NAMDCONFWDEF{wrapWater}{wrap water coordinates around periodic boundaries?}{on or off}{off}
{Coordinates are normally output relative to the way they were read in.  Hence, if part of a molecule crosses a periodic boundary it is not translated to the other side of the cell.  This option alters this behavior for water molecules only.}

\end{itemize}

\subsection{Pressure Control}

\subsubsection{Berendsen pressure bath coupling}

\NAMD\ provides constant pressure simulation using Berendsen's method.  
The following parameters are used to define the algorithm.  

\begin{itemize}

\item
\NAMDCONFWDEF{BerendsenPressure}{use Berendsen pressure bath coupling?}{\verb!on! or \verb!off!}{\verb!off!}
{Specifies whether or not Berendsen pressure bath coupling is active.  
If set to \verb!on!, then the parameters \verb!BerendsenPressureTarget!, \verb!BerendsenPressureCompressibility! and \verb!BerendsenPressureRelaxationTime! must be set 
and the parameter \verb!BerendsenPressureFreq! can
optionally be set to control the behavior of this feature.} 

\item
\NAMDCONF{BerendsenPressureTarget}{target pressure (bar)}{positive decimal}
{Specifies target pressure for Berendsen's method.}

\item
\NAMDCONF{BerendsenPressureCompressibility}{compressibility (bar$^{-1}$)}{positive decimal}
{Specifies compressibility for Berendsen's method.}

\item
\NAMDCONF{BerendsenPressureRelaxationTime}{relaxation time (fs)}{positive decimal}
{Specifies relaxation time for Berendsen's method.}

\item
\NAMDCONFWDEF{BerendsenPressureFreq}{how often to rescale positions}{positive integer}{1}
{Specifies number of timesteps between position rescalings for Berendsen's method.}

\end{itemize}

\subsubsection{Nos\'{e}-Hoover Langevin piston pressure control}

\NAMD\ provides constant pressure simulation using a modified Hos\'{e}-Hoover method in which Langevin dynamics is used to control fluctuations in the barostat.
This method should be combined with a method of temperature control, such as Langevin dynamics, in order to simulate the NPT ensemble.
The following parameters are used to define the algorithm.  

\begin{itemize}

\item
\NAMDCONFWDEF{LangevinPiston}{use Langevin piston pressure control?}{\verb!on! or \verb!off!}{\verb!off!}
{Specifies whether or not Langevin piston pressure control is active.  
If set to \verb!on!, then the parameters \verb!LangevinPistonTarget!, \verb!LangevinPistonPeriod!, \verb!LangevinPistonDecay! and \verb!LangevinPistonTemp! must be set.}

\item
\NAMDCONF{LangevinPistonTarget}{target pressure (bar)}{positive decimal}
{Specifies target pressure for Langevin piston method.}

\item
\NAMDCONF{LangevinPistonPeriod}{oscillation period (fs)}{positive decimal}
{Specifies barostat oscillation time scale for Langevin piston method.}

\item
\NAMDCONF{LangevinPistonDecay}{damping time scale (fs)}{positive decimal}
{Specifies barostat damping time scale for Langevin piston method.}

\item
\NAMDCONF{LangevinPistonTemp}{noise temperature (K)}{positive decimal}
{Specifies barostat noise temperature for Langevin piston method.
This should be set equal to the target temperature for the chosen method of temperature control.}

\end{itemize}

\subsection{Applied Forces and Analysis}

Currently, there are two ways to steer simulations with \NAMD. One
is called ``moving constraints'' and is based on the harmonic
constraints feature. The other is a stand-alone feature called
``SMD''. In both cases the user specifies a reference position 
$\vec r_0$, force constant $k$, velocity of the reference position
movement $\vec v$ and the number of the constrained atom $i$. Then
during the simulation a potential $U_{SMD}$ and a corresponding force
$\vec f_{SMD}$ are added for the specified atom. 
The potential is computed according to
\begin{equation}
   U_{SMD}(t) \; = \; \frac{k \, [\vec r(t) \, - \, \vec r_i(t)]^2}{2} \, ,
\end{equation}
and the force acting on atom $i$ is (by differentiating $U_{SMD}$) 
\begin{equation}
   \vec f_{SMD}(t) \; = \; \, k \, [\vec r(t) \, - \, \vec r_i(t)] \, ,
\label{eq:smdforce}
\end{equation}
where t is time, $\vec r_i$ is the position of the atom $i$ and 
$\vec r$ is the current reference position (position of the restraint
point), given by 
\begin{equation}
   \vec r(t) \; = \; \vec r_0 \, + \, \vec v t \,.
\label{eq:smdrefpos}
\end{equation}

\subsubsection{Moving Constraints}

Moving constraints feature works in conjunction with the Harmonic
Constraints (see an appropriate section of the User's guide).  In
addition to the usual parameters used for harmonic constraints, one
should specify the number (0-based indexing) of the constrained atom
({\tt movingConsAtom}), the reference position of which (specified in
the usual manner) will move according to Eq.~\ref{eq:smdrefpos}. Also
a velocity vector $\vec v$ ({\tt movingConsVel}) needs to be specified.

The way the moving constraints work is that the moving reference
position is calculated every integration time step using
Eq.~\ref{eq:smdrefpos}, where $\vec v$ is in \AA/timestep, and $t$ is the
current timestep (i.e., {\tt firstTimestep} plus however many
timesteps have passed since the beginning of \NAMD\ run). Therefore,
one should be careful when restarting simulations to appropriately
update the {\tt firstTimestep} parameter in the \NAMD\ configuration
file or the reference position specified in the reference PDB file.

\noindent {\bf NOTE: } \NAMD\ actually calculates the constraints
potential with $U = k (x-x_0)^d$ and the force with $F = d k (x-x_0)$,
where $d$ is the exponent {\tt consexp}. The result is that if one
specifies some value for the force constant $k$ in the PDB file,
effectively, the force constant is $2 k$ in calculations. This caveat
was removed in SMD feature.

The following parameters describe the parameters for the
moving harmonic constraint feature of \NAMD.
This feature allows a harmonic restraining force to be applied
to any ONE atom in the simulation and the restraining reference
position to be moved.

\begin{itemize}

\item
\NAMDCONFWDEF{movingConstraints}{Are moving constraints active}
{\verb!on! or \verb!off!}{\verb!off!}                                           {Should moving restraints be applied to the system. If set
to \verb!on!, then  \verb!movingConsVel! and \verb!movingConsAtom!              must be defined.}

\item
\NAMDCONF{movingConsVel}{Velocity of the reference position movement}
{vector in \AA/timestep}
{The velocity of the reference position movement. Gives both absolute
value and direction}
                                                                                \item
\NAMDCONF{movingConsAtom}{Index (0-based) of the atom constrained to
the moving reference position}
{non-negative integer}
{The index of the atom constrained to the moving reference position.
If the atom is numbered $N$ in the PDB file, $N-1$ must be specified.
\NAMD\ will crash if the atom is not constrained in the
\verb!conskfile!.
}

\end{itemize}


\subsubsection{Steered Molecular Dynamics (SMD)}

The SMD feature is independent from the harmonic constraints, although it
follows the same ideas. One has to specify the force constant $k$
({\tt SMDk}), the number (1-based indexing) of the atom ({\tt
SMDAtom}), which is restrained to the moving reference position, the
initial reference position $\vec r_0$ ({\tt SMDRefPos}), the {\em
absolute} value of the velocity of movement $v$ ({\tt SMDVel}), and
the direction of movement $\vec n$ ({\tt SMDDir}). The velocity $\vec
v$ is then given by $\vec v \, = \, v \vec n$. Vector $\vec n$ is
normalized by \NAMD\ before being used. Optionally, the frequency of
SMD data output can be specified. 

The time used in the reference position calculation starts at time
$t_0$ which can be specified through {\tt SMDTStamp} parameter,
defaulting to {\tt firstTimestep}. The reference position is
calculated then by 
\begin{equation}
  \vec r \; = \; \vec r_0 \, + \, \vec n v \, (t - t_0) \, ,
\end{equation}
where $t$ is the current timestep (i.e., {\tt firstTimestep} plus
however many timesteps have passed since the beginning of \NAMD\ run).
When restarting the simulation it may be useful to set $t_0$ to 
whatever time the reference position specified in the configuration
file refers to.

\paragraph*{Changing direction of reference position movement}

One may want to change the direction of the reference position
movement, e.g., if the restrained atom is moving too slow in the given
direction. To check for such slow movement, the minimum allowed
average velocity $v_{min}$ ({\tt SMDVmin}) and the averaging time
$\tau_{min}$ ({\tt SMDVminTave}) have to be specified. Then every
$\tau_{min}$ timesteps \NAMD\ will compute the average velocity in the
current direction over the past $\tau_{min}$ timesteps, and compare it
to $v_{min}$. The average velocity is computed simply by 
\begin{equation}
   \langle v\rangle_{min} \; = \; \frac{\vec n \cdot [\vec r_i(t) \, - \,
   	\vec r_i(t - \tau_{min}) ]}{\tau_{min}} \, .
\label{eq:smdvmin}
\end{equation}
In the case that $\langle v\rangle_{min}$ is less than $v_{min}$, 
a new direction is chosen, and the simulation proceeds. 

Every time the direction is changed, the reference point is also
changed to
\begin{equation}
  \vec r_{0,new} \; = \; \vec r_i(t) + \vec n_{new} F_{min}/k \, ,
\label{eq:smdreset}
\end{equation}
where $t$ is the current timestep, and $F_{min}$ ({\tt SMDFmin}) is
the minimum force to which the force imposed by the restraint is
reset. Thus, after the direction is changed to $\vec n_{new}$ the
force applied to the atom has an absolute value of $F_{min}$ and the
direction of $\vec n_{new}$. The time stamp $t_0$ is set to current
time.


\paragraph*{Choice of new direction}
The choice of the new (random) direction is based on the specified
initial direction $\vec n$, the angle $\theta_c$ of the cone which has
$\vec n$ as its axis, and the method by which the random numbers are
generated.  The new direction is guaranteed to lie within the
specified cone. The method can be ``uniform'' (default) or
``gaussian''. Uniform method uses uniformly distributed random numbers
to find angles $\varphi$ and $\theta$ defining the new
direction. Angle $\varphi$ is a random number between 0 and 360
degrees, and $\theta$ is a random number between 0 and
$\theta_c/2$. Gaussian method uses uniformly distributed random
numbers to find $\varphi$ and normally distributed random numbers with
variance $\sigma$ ({\tt SMDGaussW}) to find $\theta$. All $\theta$
values that exceed $\theta_c/2$ are ignored. The distribution of
$\theta$ values is given by 
\begin{equation}
p(\theta) \, = \, \frac{1}{\sigma \sqrt{2\pi}} 
\exp(- \frac{\theta^2}{\sigma^2}) \, .
\end{equation}

After the angles $\varphi$ and $\theta$ are calculated, the new
direction is found by constructing a right orthonormal system of
coordinates with $\vec n_0$, the initial direction specified by {\tt
SMDDir} in the configuration file, being the $z$ direction.  The
other two vectors $\vec a$ and $\vec b$, orthogonal to $\vec n_0$ and
to each other are found by first producing $\vec a \, = \, \vec n_0
\times
\vec{(0, 1, 1)}$, or $\vec a \, = \, \vec n_0 \times 
\vec{(1, 1, 1)}$ in the case when $\vec n_0$ is collinear to 
$\vec{(0, 1, 1)}$, normalizing $\vec a$, and, finally, producing 
$\vec b \, = \, \vec n_0 \times \vec a$. Then the new direction is
given by 
\begin{equation}
\vec n_{new} \; = \; \vec a \cos\varphi \sin\theta \, + \,
	\vec b \sin\varphi \sin\theta \, + \,
	\vec n_0 \cos\theta \, .
\end{equation}


\paragraph*{Resetting the applied force}

Sometimes, especially when a small enough $k$ is employed, the
movement of the restrained atom proceeds in steps, resulting in a fast
movement of the atom over a short period of time. When this happens,
it may be desirable to reset (reduce) the applied force. Resetting of
the force to a specified value $F_{min}$ ({\tt SMDFmin}) should happen
when the average velocity of atom movement $\langle v\rangle_{max}$
over a given period of time $\tau_{max}$ ({\tt SMDVmaxTave}) exceeds
the specified maximum allowed velocity $v_{max}$ ({\tt SMDVmax}) in
the direction of reference position movement $\vec n$. The average
velocity is calculated similarly to Eq.~\ref{eq:smdvmin}, and in the
case when $\langle v\rangle_{max}$ is larger than $v_{max}$, the
reference position is reset according to Eq.~\ref{eq:smdreset}. The
time stamp $t_0$ is set to current time.


\paragraph*{Output}

\NAMD\ provides output of the current SMD data. The frequency of
output is specified by the \verb!SMDOutputFreq! parameter in the
configuration file. Every \verb!SMDOutputFreq! timesteps \NAMD\ will
print the current timestep, current position of the restrained atom,
the current force applied to the restrained atom (in piconewtons, pN),
the reference position at the time of the last change of
direction or force resetting (initially, the reference position
specified in the configuration file), current direction of the
reference position movement, and the time stamp, indicating the
timestep when the direction was changed or force reset (initially, 
\verb!SMDTStep! specified in the configuration file, or its default
value). The output line starts with word {\tt SMD}

The energy of the SMD restraint is printed out along with other
energies in the column titled {\tt SMD}. The frequency of this output
is defined by \verb!outputEnergies! parameter in the configuration
file. 

When the direction of the reference position movement changes or the
force is reset, this change is indicated by an output line starting
with {\tt SMDChange} and showing the current timestep, the current
atom position, the new reference position after the change, and the
new direction of the reference position movement. The current timestep
becomes the time stamp for the following simulation until the next
change occurs.

\paragraph*{Parameters}

The following parameters describe the parameters for the 
SMD feature of \NAMD.
This feature allows a harmonic restraining force to be applied
to any ONE atom in the simulation and the restraining reference
position to be moved. It also allows to randomly change the direction
of the reference position movement if the restrained atom progress is
too slow, and to reset the applied force to a given value if the
progress of the restrained atom is too fast.

\begin{itemize}

\item 
\NAMDCONFWDEF{SMD}{Are SMD features active}
{\verb!on! or \verb!off!}{\verb!off!}
{Should SMD harmonic constraint be applied to the system. If set 
to \verb!on!, then  \verb!SMDk!, \verb!SMDRefPos!, \verb!SMDVel!,
\verb!SMDDir! and \verb!SMDAtom! must be defined. Also,
\verb!SMDOutputFreq!, \verb!SMDChDir! and associated parameters,
\verb!SMDChForce! and associated parameters, and \verb!SMDTStamp! can
be optionally defined.}

\item
\NAMDCONFWDEF{SMDexp}{exponent to use in SMD harmonic constraint energy 
function}{positive, even integer}{2}
{Exponent to be used in the SMD harmonic constraint energy function.
This value must be a positive integer, and only even values really make
sense.  This parameter is only used if \verb!SMD! is set to 
\verb!on!.}

\item
\NAMDCONF{SMDRefPos}{SMD constraint reference position}
{vector} {Vector to use for the initial reference position for the SMD
harmonic constraints. The atom that is specified by \verb!SMDAtom!
will be initially constrained to this reference position. During the
simulation, this reference position will move with velocity
\verb!SMDVel! in the direction \verb!SMDDir!.}

\item
\NAMDCONF{SMDk}{force constant to use in SMD simulation}
{positive real}
{SMD harmonic constraint force constant. Must be specified in
kcal/mol/\AA$^2$. The conversion factor is 1 kcal/mol = 69.479 pN \AA.} 

\item
\NAMDCONF{SMDVel}{Velocity of the SMD reference position movement}
{positive real, \AA/timestep}
{The velocity of the SMD reference position movement. Gives the absolute
value.}

\item
\NAMDCONF{SMDDir}{Direction of the SMD reference position movement}
{non-zero vector}
{The direction of the SMD reference position movement. The vector does
not have to be normalized, it is normalized by \NAMD before being used.}

\item
\NAMDCONF{SMDAtom}{Index (1-based) of the atom constrained to
the moving reference position}
{positive integer}
{The index of the atom constrained to the moving reference position. 
If the atom is numbered $N$ in the PDB file, $N$ must be specified.
}

\item
\NAMDCONFWDEF{SMDOutputFreq}{frequency of SMD output}
{positive integer}{1} {The frequency in timesteps with which the
current SMD data values are printed out.}

\item
\NAMDCONFWDEF{SMDTStamp}{time of the last change in reference
position}{non-negative integer}{\verb!firstTimestep!}
{The timestep at which the reference position $\vec r_0$ was last
reset. Normally, it would be the \verb!firstTimestep!, but if the
force or direction was reset during the simulation, this parameter may
be useful for restarts.}

\item
\NAMDCONFWDEF{SMDFmin}{value of the force to reset to}{non-negative
real}{0} {The value in pN to which the force gets reset when the
direction changes or the average velocity exceeds the given
\verb!SMDVmax! value. This parameter is only used if \verb!SMDChDir!
or \verb!SMDChForce! are
\verb!on!.}

\item
\NAMDCONFWDEF{SMDChDir}{Is changing direction option of SMD active}
{\verb!on!, \verb!off!}{\verb!off!}
{If turned \verb!on!, this option allows to change the direction of
the reference position movement when the average velocity of the
restrained atom in this direction is smaller than the given
\verb!SMDVmin! value. Uses settings of 
\verb!SMDVmin!, \verb!SMDVminTave!, \verb!SMDConeAngle!,
\verb!SMDChDirMethod!.}

\item
\NAMDCONF{SMDVmin}{minimum allowed average velocity of the restrained
atom}{non-negative real} {The minimum allowed average velocity in
\AA/timestep of the restrained atom in the direction of the reference
position movement. The averaging time is given by \verb!SMDVminTave!.}

\item
\NAMDCONF{SMDVminTave}{averaging time for velocity to compare to Vmin}
{positive integer} {The averaging time in timesteps for calculation of
the average velocity of the restrained atom and for comparison of this
average velocity to
\verb!SMDVmin!.}

\item
\NAMDCONFWDEF{SMDConeAngle}{angle of the cone to choose the new
direction from}{non-negative}{360}{The angle (in degrees) of the cone
with the axis along the initial direction of the reference position
movement \verb!SMDDir!. The new direction will come from the vertex of
the cone and will lie within the cone. This angle is twice the angle
between the cone's axis and any line going through the cone's vertex
on the cone's surface.  Omitting this parameter will lead to choosing
a direction without any restrictions (all directions are possible).}

\item
\NAMDCONFWDEF{SMDChDirMethod}{method to use when choosing a new
direction}{\verb!gaussian!, \verb!uniform!}{\verb!uniform!}
{The method to choose the angle between the new direction and the
initial direction of the reference position movement. When
\verb!gaussian! is specified, it is recommended to set the variance of
the distribution with \verb!SMDGaussW!.}

\item
\NAMDCONFWDEF{SMDGaussW}{variance of the gaussian distribution used to
choose new directions}{positive real}{360}
{Variance of the gaussian distribution (in degrees) used
for choosing the angle between the new direction and the
initial direction of the reference position movement.}


\item
\NAMDCONFWDEF{SMDChForce}{is resetting force SMD option active}
{\verb!on!, \verb!off!}{\verb!off!}
{If turned \verb!on!, this option allows to reset
the reference position (so that the applied force becomes
\verb!SMDFmin!) when the average velocity of the
restrained atom in the direction of the reference position movement is
larger than a given \verb!SMDVmax! value. Uses settings of 
\verb!SMDVmax! and \verb!SMDVmaxTave!.}

\item
\NAMDCONF{SMDVmax}{maximum allowed average velocity of the restrained
atom}{non-negative real} {The maximum allowed average velocity in
\AA/timestep of the restrained atom in the direction of the reference
position movement. The averaging time is given by \verb!SMDVmaxTave!.}

\item
\NAMDCONF{SMDVmaxTave}{averaging time for velocity to compare to Vmax}
{positive integer} {The averaging time in timesteps for calculation of
the average velocity of the restrained atom and for comparison of this
average velocity to
\verb!SMDVmax!.}
\end{itemize}

\subsubsection{Tcl interface}

\NAMD\ provides a limited Tcl scripting interface designed for applying forces and performing on-the-fly analysis.
This interface is efficient if only a few coordinates, either of individual atoms or centers of mass of groups of atoms, are needed.
In addition, information must be requested one timestep in advance.
The following configuration parameters are used to enable the Tcl interface:

\begin{itemize}

\item
\NAMDCONFWDEF{tclForces}{is Tcl interface active?}{\verb!on! or \verb!off!}{\verb!off!}
{Specifies whether or not Tcl interface is active.  If it 
is set to \verb!off!, then no Tcl code is executed.  
If it is set to \verb!on!, then Tcl code specified in
\verb!tclForcesScript! parameters is executed.}

\item
\NAMDCONF{tclForcesScript}{input for Tcl interface}{file or \{script\}}
{Must contain either the name of a Tcl script file or the script 
itself between \{ and \} (may include multiple lines).
This parameter may occur multiple times and scripts will be executed
in order of appearance.
The script(s) should perform any required initialization on the Tcl interpreter, including requesting data needed during the first timestep, and define a procedure \verb!calcforces \{ \}! to be called every timestep.
}

\end{itemize}

At this point only low-level commands are defined.
In the future this list will be expanded.  Current commands are:

\begin{itemize}

\item
\verb!print <anything>! \\
This command should be used instead of \verb!puts! to display output.
For example, ``\verb&print Hello World&''.

\item
\verb!atomid <molname> <resid> <atomname>! \\
Determines atomid of an atom from its molecule, residue, and name.
For example, ``\verb!atomid br 2 N!''.

\item
\verb!addatom <atomid>! \\
Request coordinates of this atom for next force evaluation.
Request remains in effect until \verb!reconfig! is called.
For example, ``\verb!addatom 4!'' or ``\verb!addatom [atomid br 2 N]!''.

\item
\verb!addgroup <atomid list>! \\
Request center of mass coordinates of this group for next force evaluation.
Returns a group ID which is of the form \verb!gN! where \verb!N! is a small integer.
This group ID may then be used to find coordinates and apply forces just like a regular atom ID.
Aggregate forces may then be applied to the group as whole.
Request remains in effect until \verb!reconfig! is called.
For example, ``\verb!set groupid [addgroup { 14 10 12 }]!''.

\item
\verb!reconfig! \\
Signals that new atoms are being requested.
\verb!addatom! and \verb!addgroup! calls during \verb!calcforces! will be ignored unless \verb!reconfig! is called.
Old configuration is replaced by new configuration.
\verb!reconfig! should only be called from within the \verb!calcforces! procedure.

\item
\verb!loadcoords <varname>! \\
Loads requested atom and group coordinates (in \AA) into a local array.
\verb!loadcoords! should only be called from within the \verb!calcforces! procedure.
For example, ``\verb!loadcoords p!'' and ``\verb!print p(4)!''.

\item
\verb!loadmasses <varname>! \\
Loads requested atom and group masses (in amu) into a local array.
\verb!loadmasses! should only be called from within the \verb!calcforces! procedure.
For example, ``\verb!loadcoords m!'' and ``\verb!print m(4)!''.

\item
\verb!addforce <atomid|groupid> <force vector>! \\
Applies force (in kcal mol$^{-1}$ \AA$^{-1}$) to atom or group.
\verb!addforce! should only be called from within the \verb!calcforces! procedure.
For example, ``\verb!addforce $groupid { 1. 0. 2. }!''.

\end{itemize}

Several vector routines from the VMD Tcl interface are also defined.

