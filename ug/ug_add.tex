\section{Additional Simulation Parameters}
\label{section:add}

\subsection{\NAMD\ configuration parameters}
\label{section:config_add}

\paragraph{Harmonic constraints}

The following describes the parameters for the 
harmonic constraints feature of \NAMD.  Actually, this feature 
should be referred to as harmonic restraints rather than 
constraints, but for historical reasons the terminology of 
harmonic constraints has been carried over from X-PLOR.  
This feature allows a harmonic restraining force to be applied 
to any set of atoms in the simulation.  For further details, 
see the \PG.  

\begin{itemize}

\item
\NAMDCONFWDEF{constraints}{are constraints active?}{\verb!on! or \verb!off!}{\verb!off!}
{Specifies whether or not harmonic constraints are active.  If it 
is set to \verb!off!, then no harmonic constraints are computed.  
If it is set to \verb!on!, then 
harmonic constraints are calculated using the values specified 
by the parameters \verb!consref!, \verb!conskfile!, \verb!conskcol!, 
and \verb!consexp!.}

\item
\NAMDCONFWDEF{consexp}{exponent for harmonic constraint energy function}{positive, even integer}{2}
{Exponent to be use in the harmonic constraint energy function.  
This value must be a positive integer, and only even values really make 
sense.  This parameter is used only if \verb!constraints! is set to 
\verb!on!.}

\item
\NAMDCONFWDEF{consref}{PDB file containing constraint reference positions}{UNIX file name}{\verb!coordinates!}
{PDB file to use for reference positions for harmonic constraints.  
Each atom that has an active constraint will be constrained about 
the position specified in this file.  If no value is given and constraints 
are active, then the same PDB file specified by \verb!coordinates! will be 
used instead, constraining atoms about their initial positions.}

\item
\NAMDCONFWDEF{conskfile}{PDB file containing force constant values}{UNIX filename} {\verb!coordinates!}
{PDB file to use for force constants for 
harmonic constraints.  
If this parameter is not specified, then 
the PDB file containing initial coordinates specified by 
\verb!coordinates! is used.}

\item
\NAMDCONFWDEF{conskcol}{column of PDB file containing force constant}{\verb!X!, \verb!Y!, \verb!Z!, \verb!O!, or \verb!B!}{\verb!O!}
{Column of the PDB file to use for the harmonic constraint force constant.
This parameter may specify any of the floating point fields of the PDB file, 
either X, Y, Z, occupancy, or beta-coupling (temperature-coupling).  
Regardless of which column is used, a value of 0 indicates that the atom 
should not be constrained.  
Otherwise, the value specified is used as the force constant for 
that atom's restraining potential.}

\end{itemize}

\paragraph{Energy minimization parameters}

As described in the \PG, \NAMD\ has the capability
of performing energy minimization using a simple quenching
scheme.   While this algorithm is not the most rapidly convergent, it
is sufficient for most applications.  There are only two parameters
for minimization:  one to activate minimization and another
to specify the maximum movement of any atom.  

\begin{itemize}

\item
\NAMDCONFWDEF{minimization}{Perform energy minimization?}{\verb!on! or \verb!off!}{\verb!off!}
{Turns energy minimization \verb!on! or \verb!off!.}

\item
\NAMDCONFWDEF{maximumMove}{maximum distance an atom can move during each step (\AA)}%
{positive decimal}%
{$0.75\times\mbox{\verb!cutoff!}/\mbox{\verb!stepsPerCycle!}$}
{Maximum distance that an atom can move during any single timestep of
minimization.  This is to insure that atoms do not go flying off into
space during the first few timesteps when the largest energy conflicts
are resolved.}

\end{itemize}

\paragraph{Langevin dynamics parameters}

As described in the \PG, \NAMD\ is capable
of performing Langevin dynamics, where additional damping and
random forces are introduced to the system.  This capability
is based on that implemented in X-PLOR which is detailed
in the X-PLOR {\it User's Manual} \mycite{(Br\"unger, 1992)}{BRUN92b}.

\begin{itemize}

\item
\NAMDCONFWDEF{langevin}{use Langevin dynamics?}{\verb!on! or \verb!off!}{\verb!off!}
{Specifies whether or not Langevin dynamics active.  
If set to \verb!on!, then the parameter \verb!langevinTemp! must be set 
and the parameters \verb!langevinFile! and \verb!langevinCol! can
optionally be set to control the behavior of this feature.} 

\item
\NAMDCONF{langevinTemp}{temperature for Langevin calculations (K)}{positive decimal}
{Temperature to which atoms affected by Langevin dynamics will be adjusted.  
This temperature will be roughly maintained across the affected atoms 
through the addition of friction and random forces.}

\item
\NAMDCONFWDEF{langevinFile}{PDB file containing Langevin parameters}%
{UNIX filename}{\verb!coordinates!}
{PDB file to use for the Langevin coupling coefficients for each atom.  
If this parameter is not specified, then 
the PDB file specified by \verb!coordinates! is used.}

\item
\NAMDCONFWDEF{langevinCol}{column of PDB from which to read coefficients}
{\verb!X!, \verb!Y!, \verb!Z!, \verb!O!, or \verb!B!}{\verb!O!} 
{Column of the PDB file to use for the Langevin coupling coefficients for 
each atom.  The coefficients can be read from any 
floating point column of the PDB file.  
A value of 0 indicates that the atom will remain unaffected.}

\end{itemize}

%% \newpage
%% \paragraph{Dihedral angle dynamics parameters}
%% 
%% \begin{itemize}
%% 
%% \item
%% \NAMDCONFWDEF{dihedral}{Should dihedral angle dynamics be calculated?}{\verb!yes! or \verb!no!}{\verb!no!}
%% {This parameter determines whether or not dihedral angle dynamics
%% are calculated.  If set to \verb!yes!, then the optional parameters
%% \verb!COLD!, \verb!COLDTemp!, and \verb!COLDRate!
%% may be specified.}
%% 
%% \item
%% \NAMDCONFWDEF{COLD}{Should overdamped Langevin dynamics be performed?}{\verb!yes! or \verb!no!}{\verb!no!}
%% {This parameter determines whether or not overdamped dynamics calculations
%% are made.  If set to \verb!yes!, then the additional parameters
%% \verb!COLDTemp! and \verb!COLDRate! are required.}
%% 
%% \item
%% \NAMDCONF{COLDTemp}{Temperature for heat bath used in COLD calculations (K)}{Positive decimal}
%% {This parameter must be specified if \verb!COLD! is set to \verb!yes!.}
%% 
%% \item
%% \NAMDCONFWDEF{COLDRate}{Damping rate for COLD calculations}{Non-negative decimal}{\verb!3000.0!}
%% {This parameter is required if \verb!COLD! is set to \verb!yes!. If no
%% value is specified, 3000 is assumed.}
%% 
%% \end{itemize}

\paragraph{Temperature coupling parameters}

As described in the \PG, \NAMD\ is capable
of performing temperature coupling, in which forces are added or 
reduced to simulate the coupling of the system to a heat bath 
of a specified temperature.  
This capability is based on that implemented in X-PLOR which is detailed
in the X-PLOR {\it User's Manual} \mycite{(Br\"unger, 1992)}{BRUN92b}.

\begin{itemize}

\item
\NAMDCONFWDEF{tCouple}{perform temperature coupling?}{\verb!on! or \verb!off!}{\verb!off!}
{Specifies whether or not temperature coupling is active.  
If set to \verb!on!, then the parameter \verb!tCoupleTemp! must be set and 
the parameters \verb!tCoupleFile! and \verb!tCoupleCol! can 
optionally be set to control the behavior of this feature.} 

\item
\NAMDCONF{tCoupleTemp}{temperature for heat bath (K)}{positive decimal}
{Temperature to which atoms affected 
by temperature coupling will be adjusted.  
This temperature will be roughly maintained across the affected atoms 
through the addition of forces.}

\item
\NAMDCONFWDEF{tCoupleFile}{PDB file with tCouple parameters}%
{UNIX filename}{\verb!coordinates!}%
{PDB file to use for the temperature coupling coefficient for each atom.  
If this parameter is not specified, then 
the PDB file specified by \verb!coordinates! is used.} 

\item
\NAMDCONFWDEF{tCoupleCol}{column of PDB from which to read coefficients}
{\verb!X!, \verb!Y!, \verb!Z!, \verb!O!, or \verb!B!}{\verb!O!} 
{Column of the PDB file to use for the temperature coupling coefficient for 
each atom.  This value can be read from any 
floating point column of the PDB file.  
A value of $0$ indicates that the atom will remain unaffected.}

\end{itemize}

\paragraph{Temperature rescaling parameters}

\NAMD\ allows equilibration of a system by means of temperature 
rescaling.  Using this method, all of the velocities in the system 
are periodically rescaled so that the entire system is set to the 
desired temperature.  The following parameters specify how often 
and to what temperature this rescaling is performed.  

\begin{itemize}

\item
\NAMDCONF{rescaleFreq}{number of timesteps between temperature rescaling}{positive integer}
{The equilibration feature of \NAMD\ is activated by 
specifying the number of timesteps between each temperature rescaling.  
If this value is given, then the \verb!rescaleTemp! parameter must also 
be given to specify the target temperature. }

\item
\NAMDCONF{rescaleTemp}{temperature for equilibration (K)}{positive decimal}
{The temperature to which all velocities will be rescaled
every \verb!rescaleFreq! timesteps.  
This parameter is valid only if \verb!rescaleFreq! has been set.}

\end{itemize}

\paragraph{Spherical harmonic boundary conditions}

\NAMD\ provides spherical harmonic boundary conditions.  These 
boundary conditions can consist of a single potential or a 
combination of two potentials as described in the \PG.  
The following parameters are used to define these boundary conditions.  

\begin{itemize}

\item
\NAMDCONFWDEF{sphericalBC}{use spherical boundary conditions?}{\verb!on! or \verb!off!}{\verb!off!}
{Specifies whether or not spherical boundary conditions 
are to be applied to the system.  If 
set to \verb!on!, then \verb!sphericalBCr1! and \verb!sphericalBCk1! 
must be defined, and \verb!sphericalBCexp1!, \verb!sphericalBCr2!, 
\verb!sphericalBCk2!, and \verb!sphericalBCexp2! can optionally be 
defined.}

\item
\NAMDCONF{sphericalBCr1}{radius for first boundary condition (\AA)}{positive decimal}
{Distance at which the first potential of the boundary conditions takes
effect.  This distance is a radius from the center of mass.}

\item
\NAMDCONF{sphericalBCk1}{force constant for first potential}{non-zero decimal}
{Force constant for the first harmonic potential.  A positive
value will push atoms toward the center of mass, and a negative
value will pull atoms away from the center of mass.}

\item
\NAMDCONFWDEF{sphericalBCexp1}{exponent for first potential}{positive, even integer}{2}
{Exponent for first boundary potential.  The only likely values to
use are 2 and 4.}

\item
\NAMDCONF{sphericalBCr2}{radius for second boundary condition (\AA)}{positive decimal}
{Distance at which the second potential of the boundary conditions takes
effect.  This distance is a radius from the center of mass.
If this parameter is defined, then \verb!spericalBCk2! must also
be defined.}

\item
\NAMDCONF{sphericalBCk2}{force constant for second potential}{non-zero decimal}
{Force constant for the second harmonic potential.  A positive
value will push atoms toward the center of mass, and a negative
value will pull atoms away from the center of mass.}

\item
\NAMDCONFWDEF{sphericalBCexp2}{exponent for second potential}{positive, even integer}{2}
{Exponent for second boundary potential.  The only likely values to
use are 2 and 4.}

\end{itemize}

\paragraph{Applied electric field}

\NAMD\ provides the ability to apply a constant electric field to the molecular
system being simulated.  There are two parameters that control this feature.

\begin{itemize}

\item
\NAMDCONFWDEF{eFieldOn}{apply electric field?}{\verb!yes! or \verb!no!}{\verb!no!}
{Specifies whether or not an electric field is applied.}

\item
\NAMDCONF{eField}{electric field vector}{vector of decimals}
{Vector which describes the electric field to be applied.}

\end{itemize}

