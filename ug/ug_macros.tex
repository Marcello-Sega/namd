%%%%%%%%%%%%%%%%%%%%%%%%%%%%%%%%%%%%%%%%%%%%%%%%%%%%%%%%%%%%%%%%%%%%%%%%%%%%
%                                                                          %
%              (C) Copyright 1995 The Board of Trustees of the             %
%                          University of Illinois                          %
%                           All Rights Reserved                            %
%								  	   %
%%%%%%%%%%%%%%%%%%%%%%%%%%%%%%%%%%%%%%%%%%%%%%%%%%%%%%%%%%%%%%%%%%%%%%%%%%%%

\newcommand{\DOCTITLE} {User's Guide}
\newcommand{\DOCDESC} {%
The \NAMD\ {\em\DOCTITLE\/} describes how to run and use the 
various features of the molecular dynamics program \NAMD.  
This guide includes the capabilities of the program, how 
to use these capabilities, the necessary input files and 
formats, and how to run the program both on uniprocessor 
machines and in parallel.}
\newcommand{\PG}{\NAMD\ {\it Programmer's Guide\/}}
\newcommand{\UG}{\NAMD\ {\it User's Guide\/}}
\newcommand{\prettypar}{

\smallskip

}

%%%%%%%%%%%%%%%%%%%%%%%%%%%%%%%%%%%%%%%%%%%%%%%%%%%%%%%%%%
% GENERAL STUFF
%%%%%%%%%%%%%%%%%%%%%%%%%%%%%%%%%%%%%%%%%%%%%%%%%%%%%%%%%%

\newcommand{\eg}{{\it e.g.\/}}
\newcommand{\ie}{{\it i.e.\/}}
\newcommand{\viz}{{\it viz. }}
\newcommand{\ca}{{\it ca. }}
\newcommand{\cf}{{\it cf. }}

\newcommand{\on}{{\tt on}}
\newcommand{\off}{{\tt off}}
\newcommand{\yes}{{\tt yes}}
\newcommand{\no}{{\tt no}}
\newcommand{\none}{{\tt none}}

\newcommand{\tcl}{Tcl}
\newcommand{\unix}{Unix}

\newcommand{\namd}{\NAMD}
\newcommand{\charmm}{CHARMM}

\newcommand{\deca}{deca-{\sc l}-alanine}



%%%%%%%%%%%%%%%%%%%%%%%%%%%%%%%%%%%%%%%%%%%%%%%%%%%%%%%%%%
% FORMULAE
%%%%%%%%%%%%%%%%%%%%%%%%%%%%%%%%%%%%%%%%%%%%%%%%%%%%%%%%%%

\newcommand{\dd}{{\rm d}}
\newcommand{\dr}{\partial}

\newcommand{\bsigma}{\mbox{\boldmath$\varsigma$}}

\newcommand{\pot}{{\mathcal V}({\bf x})}
\newcommand{\potk}{{\mathcal V}({\bf x}; \lambda_k)}
\newcommand{\potl}{{\mathcal V}({\bf x}; \lambda_{k+1})}

\newcommand{\potx}{{\mathcal V}({\bf x}; \xi)}

\newcommand{\potq}{{\mathcal V}({\bf q}, \xi)}
\newcommand{\potqs}{{\mathcal V}({\bf q}, \xi^*)}
\newcommand{\kin}{{\mathcal T}({\bf p}_x)}
\newcommand{\ham}{{\mathcal H}({\bf x}, {\bf p}_x)}

\newcommand{\pxi}{{\mathcal P}_\xi}

\newcommand{\gradx}{\mbox{\boldmath$\nabla_{\!\!x}\,$}}




\newcommand{\KEY}[1]{{\tt #1}}
\newcommand{\IKEY}[1]{{\tt #1\index{#1 psfgen command}}}
\newcommand{\OKEY}[1]{$[${\tt #1}$]$}
\newcommand{\ARG}[1]{$<${\em #1}$>$}
\newcommand{\OARG}[1]{$[${\em #1}$]$}
\newcommand{\ARGDEF}[2]{$<${\em #1}$>$: #2}
\newcommand{\KEYDEF}[2]{{\tt #1}: #2}
\newcommand{\COMMAND}[4]{%
  #1 \\ {\bf Purpose:} #2 \\ {\bf Arguments:} #3 \\ {\bf Context:} #4 }

\newcommand{\icommand}[1]{#1\index{#1 command}}

\newcommand{\NAMDCONF}[4]{%
%  \addcontentsline{toc}{subparagraph}{#1}%
  {\bf \tt #1 } $<$ #2 $>$ \index{#1 parameter} \\%
  {\bf Acceptable Values: } #3 \\%
  {\bf Description: } #4%
}

\newcommand{\NAMDCONFWDEF}[5]{%
%  \addcontentsline{toc}{subparagraph}{#1}%
  {\bf \tt #1 } $<$ #2 $>$ \index{#1 parameter} \\%
  {\bf Acceptable Values: } #3 \\%
  {\bf Default Value: } #4 \\%
  {\bf Description: } #5%
}

\newcommand{\XNCOMP}[3]{%
  {\bf \NAMD\ Parameter: \tt #1 } \\%
  {\bf X-PLOR Parameter: \tt #2 } \\%
  #3%
}

