
\section{Accelerated Sampling Methods}
\label{section:accel}


\subsection{Locally enhanced sampling}
\label{section:les}

Locally enhanced sampling (LES)~\cite{ROIT91,SIMM98,SIMM00} increases
sampling and transition rates for a portion of a molecule by the use of
multiple non-interacting copies of the enhanced atoms.  These enhanced
atoms experience an interaction (electrostatics, van der Waals, and
covalent) potential that is divided by the number of copies present.
In this way the enhanced atoms can occupy the same space, while the
multiple instances and reduces barriers increase transition rates.

\subsubsection{Structure generation}

To use LES, the structure and coordinate input files must be modified to
contain multiple copies of the enhanced atoms.  \PSFGEN\ provides the
{\tt multiply} command for this purpose.  \NAMD\ supports a maximum of 15
copies, which should be sufficient.  

Begin by generating the complete molecular structure and guessing
coordinates as described in Sec.~\ref{section:psfgen}.  As the last
operation in your script, prior to writing the psf and pdb files, add
the {\tt multiply} command, specifying the number of copies desired and
listing segments, residues, or atoms to be multiplied.  For example,
\verb#multiply 4 BPTI:56 BPTI:57# will create four copies of the last
two residues of segment BPTI.  You must include all atoms to be
enhanced in a single {\tt multiply} command in order for the bonded
terms in the psf file to be duplicated correctly.  Calling {\tt multiply}
on connected sets of atoms multiple times will produce unpredictable
results, as may running other commands after {\tt multiply}.

The enhanced atoms are duplicated exactly in the structure---they have
the same segment, residue, and atom names.  They are distinguished only
by the value of the B (beta) column in the pdb file, which is 0 for
normal atoms and varies from 1 to the number of copies created for
enhanced atoms.  The enhanced atoms may be easily observed in VMD with
the atom selection \verb#beta != 0#.

\subsubsection{Simulation}

In practice, LES is a simple method used to increase sampling;
no special output is generated.
The following parameters are used to enable LES:

\begin{itemize}

\item
\NAMDCONFWDEF{les}{is locally enhanced sampling active?}{{\tt on} or {\tt
off}}{{\tt off}}
{Specifies whether or not LES is active.}

\NAMDCONF{lesFactor}{number of LES images to use}
{positive integer equal to the number of images present}
{This should be equal to the factor used in {\tt multiply}
 when creating the structure.  The interaction potentials for images is
 divided by {\tt lesFactor}.  
}

\item
\NAMDCONFWDEF{lesReduceTemp}{reduce enhanced atom temperature?}{{\tt on} or {\tt
off}}{{\tt off}}
{Enhanced atoms experience interaction potentials divided by {\tt lesFactor}.
This allows them to enter regions that would not normally be thermally
accessible.  If this is not desired, then the temperature of these atoms
may be reduced to correspond with the reduced potential.  This option
affects velocity initialization, reinititialization, reassignment, and
the target temperature for langevin dynamics.  Langevin dynamics is
recommended with this option, since in a constant energy simulation energy
will flow into the enhanced degrees of freedom until they reach thermal
equilibrium with the rest of the system.  The reduced temperature atoms
will have reduced velocities as well, unless {\tt lesReduceMass} is also
enabled.}

\item
\NAMDCONFWDEF{lesReduceMass}{reduce enhanced atom mass?}{{\tt on} or {\tt off}}{{\tt off}}
{Used with {\tt lesReduceTemp} to restore velocity distribution to
enhanced atoms.  If used alone, enhanced atoms would move faster than
normal atoms, and hence a smaller timestep would be required.}

\item
\item
\NAMDCONFWDEF{lesFile}{PDB file containing LES flags}{UNIX filename} {{\tt coordinates}}
{PDB file to specify the LES image number of each atom.
If this parameter is not specified, then 
the PDB file containing initial coordinates specified by 
{\tt coordinates} is used.}

\item
\NAMDCONFWDEF{lesCol}{column of PDB file containing LES flags}{{\tt X}, {\tt Y}, {\tt Z}, {\tt O}, or {\tt B}}{{\tt B}}
{Column of the PDB file to specify the LES image number of each atom.
This parameter may specify any of the floating point fields of the PDB file, 
either X, Y, Z, occupancy, or beta-coupling (temperature-coupling).  
A value of 0 in this column indicates that the atom is not enhanced.
Any other value should be a positive integer less than {\tt lesFactor}.}

\end{itemize}


\subsection{Replica exchange simulations}

\index{replica exchange}
The {\tt lib/replica/}
directory contains Tcl scripts that implement replica exchange
for NAMD, using a Tcl server and socket connections to drive a
separate NAMD process for every replica used in the simulation.
Replica exchanges and energies are recorded in the potenergy.dat,
realtemp.dat, and targtemp.dat files written in the output directory.
These can be viewed with, e.g., ``{\tt xmgrace -nxy ....potenergy.dat}''
There is also a script to load the output into VMD and color each
frame according to target temperature.  An example simulation folds
a 66-atom model of a deca-alanine helix in about 10\,ns.

This implementation is designed to be modified by the user to implement
exchanges of parameters other than temperature or via other temperature
exchange methods.  The scripts should provide a good starting point for
any simulation method requiring a number of loosely interacting systems.

{\tt replica\_exchange.tcl}
is the master Tcl script for replica exchange simulations, it is run in
{\tt tclsh} {\em outside of NAMD} and takes a replica exchange config
file as an argument:
\begin{verbatim}
          tclsh ../replica_exchange.tcl fold_alanin.conf
          tclsh ../replica_exchange.tcl restart_1.conf
\end{verbatim}
{\tt replica\_exchange.tcl} uses code in
{\tt namd\_replica\_server.tcl}, a general script for driving NAMD slaves, and
{\tt spawn\_namd.tcl}, a variety of methods for launching NAMD slaves.

{\tt show\_replicas.vmd} is a script for loading replicas into VMD;
first source the replica exchange conf file and then this script, then
repeat for each restart conf file or for example just do
``{\tt vmd -e load\_all.vmd}''.
This script will likely destroy anything else you are doing in VMD at the
time, so it is best to start with a fresh VMD.
{\tt clone\_reps.vmd} provides the {\tt clone\_reps} commmand to copy graphical
representation from the top molecule to all other molecules.

A replica exchange config file should define the following Tcl variables:
\begin{itemize}
\item {\tt num\_replicas}, the number of replica simulations to use,
\item {\tt min\_temp}, the lowest replica target temperature,
\item {\tt max\_temp}, the highest replica target temperature,
\item {\tt steps\_per\_run}, the number of steps between exchange attempts,
\item {\tt num\_runs}, the number of runs before stopping
(should be divisible by {\tt runs\_per\_frame} $\times$ {\tt frames\_per\_restart}).
\item {\tt runs\_per\_frame}, the number of runs between trajectory outputs,
\item {\tt frames\_per\_restart}, the number of frames between restart outputs,

\item {\tt namd\_config\_file}, the NAMD config file containing all parameters,
needed for the simulation except {\tt seed}, {\tt langevin}, 
{\tt langevinDamping}, {\tt langevinTemp}, {\tt outputEnergies},
{\tt outputname}, {\tt dcdFreq},
{\tt temperature}, {\tt bincoordinates}, {\tt binvelocities},
or {\tt extendedSystem}, which are provided by {\tt replica\_exchange.tcl},

\item {\tt output\_root}, the directory/fileroot for output files,

\item {\tt psf\_file}, the psf file for {\tt show\_replicas.vmd}, 
\item {\tt initial\_pdb\_file}, the initial coordinate pdb file for {\tt show\_replicas.vmd},
\item {\tt fit\_pdb\_file}, the coodinates that frames are fit to by {\tt show\_replicas.vmd} (e.g., a folded structure),
\item {\tt server\_port}, the port to connect to the replica server on, and
\item {\tt spawn\_namd\_command}, a command from {\tt spawn\_namd.tcl} and arguments to launch NAMD jobs.
\end{itemize}

The {\tt lib/replica/example/} directory contains
all files needed to fold a 66-atom model of a deca-alanine helix:
\begin{itemize}
\item {\tt alanin\_base.namd}, basic config options for NAMD,
\item {\tt alanin.params}, parameters,
\item {\tt alanin.psf}, structure,
\item {\tt unfolded.pdb}, initial coordinates,
\item {\tt alanin.pdb}, folded structure for fitting in {\tt show\_replicas.vmd},
\item {\tt fold\_alanin.conf}, config file for {\tt replica\_exchange.tcl} script,
\item {\tt restart\_1.conf}, config file to continue alanin folding another 10\,ns, and
\item {\tt load\_all.vmd}, load all output into VMD and color by target temperature.
\end{itemize}

The {\tt fold\_alanin.conf} config file contains the following settings:
\begin{verbatim}
set num_replicas 8
set min_temp 300
set max_temp 600
set steps_per_run 1000
set num_runs 10000
set runs_per_frame 10
set frames_per_restart 10
set namd_config_file "alanin_base.namd"
set output_root "output/fold_alanin" ; # directory must exist
set psf_file "alanin.psf"
set initial_pdb_file "unfolded.pdb"
set fit_pdb_file "alanin.pdb"
set namd_bin_dir /Projects/namd2/bin/current/Linux64
set server_port 3177
set spawn_namd_command \
  [list spawn_namd_ssh "cd [pwd]; [file join $namd_bin_dir namd2] +netpoll" \
  [list beirut belfast] ]
\end{verbatim}

