%%%%%%%%%%%%%%%%%%%%%%%%%%%%%%%%%%%%%%%%%%%%%%%%%%%%%%%%%%%%%%%%%%%%%%%%%%%%
%                                                                          %
%              (C) Copyright 1995 The Board of Trustees of the             %
%                          University of Illinois                          %
%                           All Rights Reserved                            %
%								  	   %
%%%%%%%%%%%%%%%%%%%%%%%%%%%%%%%%%%%%%%%%%%%%%%%%%%%%%%%%%%%%%%%%%%%%%%%%%%%%

%%%%%%%%%%%%%%%%%%%%%%%%%%%%%%%%%%%%%%%%%%%%%%%%%%%%%%%%%%%%%%%%%%%%%%%%%%%%
% RCS INFORMATION:
%
%       $RCSfile: ug_avail.tex,v $
%       $Author: jim $        $Locker:  $                $State: Exp $
%       $Revision: 1.2 $      $Date: 1998/03/06 09:08:09 $
%
%%%%%%%%%%%%%%%%%%%%%%%%%%%%%%%%%%%%%%%%%%%%%%%%%%%%%%%%%%%%%%%%%%%%%%%%%%%%
% DESCRIPTION:
%	This section is part of the NAMD Users Guide and describes the
% the availability and installation of NAMD
%
%%%%%%%%%%%%%%%%%%%%%%%%%%%%%%%%%%%%%%%%%%%%%%%%%%%%%%%%%%%%%%%%%%%%%%%%%%%%
% REVISION HISTORY:
%
% $Log: ug_avail.tex,v $
% Revision 1.2  1998/03/06 09:08:09  jim
% Documented new features in 2.0.  Added authors.  Still needs converse docs.
%
% Revision 1.1  1998/01/05 21:12:31  dhardy
% user guide, first draft
%
% Revision 1.13  1997/08/07 17:28:20  dhardy
% minor revisions
%
% Revision 1.12  1997/08/07 15:54:32  dhardy
% edit presentation style, add control of precision during compile
%
% Revision 1.11  1997/02/25 22:35:20  dalke
% *** empty log message ***
%
% Revision 1.10  1996/06/06  18:02:01  dalke
% fixed typos for 1.4 release
%
% Revision 1.9  1996/05/15  19:21:07  jean
% ready (I hope) for 1.4 beta release
%
% Revision 1.8  1996/03/10 20:35:16  jean
% namd -> \NAMD\
%
% Revision 1.7  1996/03/10 20:11:27  jean
% minor polishing
%
% Revision 1.6  1995/10/04 14:41:39  nelson
% Fixed spelling and changed status of IBM SP1 and SP2
%
% Revision 1.5  95/06/30  15:56:19  15:56:19  gursoy (Attila Gursoy)
% added Power Challenge and SP2
% 
% Revision 1.4  95/06/30  15:49:44  15:49:44  gursoy (Attila Gursoy)
% *** empty log message ***
% 
% Revision 1.3  95/06/30  15:37:43  15:37:43  gursoy (Attila Gursoy)
% added platforms
% 
% Revision 1.2  95/06/28  23:55:46  23:55:46  gursoy (Attila Gursoy)
% first draft
% 
% Revision 1.1  95/06/19  16:08:15  16:08:15  nelson (Mark T. Nelson)
% Initial revision
% 
%%%%%%%%%%%%%%%%%%%%%%%%%%%%%%%%%%%%%%%%%%%%%%%%%%%%%%%%%%%%%%%%%%%%%%%%%%%%

\section{\NAMD\ Availability and Installation}
\label{section:avail}

\NAMD\ is distributed freely for non-profit use.  Portability 
across various architectures is achieved by using widely available 
message-passing software layers.  \NAMD\ currently runs on 
PVM, a message-passing library available for distributed and 
parallel computers.  
%%  Charm is an object-based parallel programming environment
%% available from University of Illinois (\verb+http://charm.cs.uiuc.edu+).
%% Currently, at the ftp site, only the PVM version of \NAMD\ is available.
\prettypar
This section describes how to obtain \NAMD\ and how to install the PVM based 
version.  In order to compile and run this version, PVM 3.x must be 
installed on the target machines.  

\subsection{How to obtain \NAMD}

\NAMD\ is available as the compressed tar file 
{\tt namd-xxx.tar.gz} obtained via anonymous 
FTP from the server {\tt ftp.ks.uiuc.edu} in the directory 
{\tt pub/mdscope/namd}.  
Directories from this release and their content are as follows: 
\begin{tabbing}
xxxxx \= xxxxxxxx\= \kill \\
\> {\tt DPMTA} \>
         DPMTA for HPUX-9.x from the Scientific Computing group at Duke \\
\> {\tt doc} \> 
         Documentation for \NAMD\ \\
\> {\tt mdcomm} \> 
         MDComm software, currently available only for HPUX-9.x and 
IRIX-5.x \\ 
\> {\tt src} \>
         \NAMD\ source code \\
\> {\tt sample} \>
         Sample \NAMD\ configuration file and protein
\end{tabbing}

DPMTA is the Distributed Parallel Multipole Treecode Algorithm 
developed by the scientific computing group at Duke University.  
\NAMD\ is capable of interacting with DPMTA to provide full electrostatic 
calculations.  A pre-compiled version of DPMTA for HP systems is included 
with this release for convenience.  The source code, 
which can be compiled for other platforms, 
may be obtained directly from Duke via anonymous FTP from 
the server {\tt ftp.ee.duke.edu} in the directory {\tt /pub/SciComp/src}.
\prettypar
MDComm is a library and set of daemon processes that allow \NAMD\ to 
communicate with the graphics program \VMD\ running on a remote machine 
via TCP/IP.  This library is currently available only for HPUX-9.x and 
IRIX-5.x systems.  
\prettypar
MTS is the Memory Tuning Software from New Code, Inc\. which provides a 
faster memory allocation scheme than that available from most UNIX 
implementations.  On HPUX-9.x systems, it achieves an increase in overall 
\NAMD\ performance of about 10\%.  This software is available from
NewCode, Inc.
\prettypar
If any of 
DPMTA, MDComm, or MTS are not available or desired, simply comment 
out the corresponding 
section of the appropriate \verb+Makedata.<architecture>+ file.  

\subsection{Platforms on which \NAMD\ will currently run}
\NAMD\ is expected to run on any parallel platform with a C++ compiler
and PVM version 3.3.6 or later.
Some minor tuning of system parameters or compiler flags 
may be required.  Such parameters include PVM buffer sizes, etc.  
\NAMD\ has been compiled and tested on the following platforms:  
\begin{itemize}
\item HP workstations (HPUX)
\item SGI workstations  (IRIX5)
\item IBM workstations (AIX)
\item Cray T3D
\item Convex Exemplar
\item IBM SP1 and SP2 using PVMe
\end{itemize}
%% List of machines that we plan to test soon include the SGI Power
%% Challenge and Intel Paragon.

\subsection{Compiling \NAMD}
To compile \NAMD\ for a particular machine, follow these steps:  
%%  \begin{tabbing}
%%  xxxxx \= xxxxx \= \kill \\
%%  \> 1. \> Go to the {\tt src} directory. \\
%%  \> 2. \> Edit {\tt Makefile} and change the ARCH variable to the desired 
%%           system\\ 
%%  \>    \> \eg, one of HPUX9, IRIX5, AIX, T3D, or Exemplar. \\
%%  \> 3. \> Edit the \verb+Makedata.<architecture>+ file to get the 
%%           correct options\\
%%  \>    \>  and directory specifications for your platform.  \\
%%  \> 4. \> Type \verb!make!.
%%  \end{tabbing}
\begin{enumerate}
\item 
Go to the {\tt src} directory. 
\item
Edit {\tt Makefile} and change the ARCH variable to the desired system 
\eg, one of HPUX9, IRIX5, AIX, T3D, or Exemplar.  
\item 
Edit the \verb+Makedata.<architecture>+ file to get the 
correct options and directory specifications for your platform.  
For example, precision can be controlled at compile time 
to yield single rather than the default double precision.  
The \verb!ARCH_COPTS! (C compiler options) variable 
would need to be changed 
in \verb+Makedata.<architecture>+ 
to define \verb!SHORTREALS! to the preprocessor.  
\item
Type \verb!make!.
\end{enumerate}
Pre-compiled binaries are provided for HP and SGI workstations.

\subsection{Documentation}
The directory {\tt doc} contains three documents:
\begin{itemize}
\item the \UG\ ({\tt ug.ps}, this document) describes how to use \NAMD, 
\item the \PG\ ({\tt pg.ps}) provides a complete description of \NAMD\ and 
  its implementation, 
\item {\tt configsheet.ps} provides a one page summary of the 
  configuration options for \NAMD.  
\end{itemize}

