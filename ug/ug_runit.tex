%%%%%%%%%%%%%%%%%%%%%%%%%%%%%%%%%%%%%%%%%%%%%%%%%%%%%%%%%%%%%%%%%%%%%%%%%%%%
%                                                                          %
%              (C) Copyright 1995 The Board of Trustees of the             %
%                          University of Illinois                          %
%                           All Rights Reserved                            %
%								  	   %
%%%%%%%%%%%%%%%%%%%%%%%%%%%%%%%%%%%%%%%%%%%%%%%%%%%%%%%%%%%%%%%%%%%%%%%%%%%%

%%%%%%%%%%%%%%%%%%%%%%%%%%%%%%%%%%%%%%%%%%%%%%%%%%%%%%%%%%%%%%%%%%%%%%%%%%%%
% RCS INFORMATION:
%
%       $RCSfile: ug_runit.tex,v $
%       $Author: milind $        $Locker:  $                $State: Exp $
%       $Revision: 1.2 $      $Date: 1998/03/09 23:42:15 $
%
%%%%%%%%%%%%%%%%%%%%%%%%%%%%%%%%%%%%%%%%%%%%%%%%%%%%%%%%%%%%%%%%%%%%%%%%%%%%
% DESCRIPTION:
%	This file contains the section of the NAMD User's Guide which
% describes how to actually run the program.
%
%%%%%%%%%%%%%%%%%%%%%%%%%%%%%%%%%%%%%%%%%%%%%%%%%%%%%%%%%%%%%%%%%%%%%%%%%%%%
% REVISION HISTORY:
%
% $Log: ug_runit.tex,v $
% Revision 1.2  1998/03/09 23:42:15  milind
% Rewrote the entire section removing the PVM parts, and adding the converse
% details.
%
% Revision 1.1  1998/01/05 21:12:34  dhardy
% user guide, first draft
%
% Revision 1.8  1997/08/07 17:28:20  dhardy
% minor revisions
%
% Revision 1.7  1997/08/07 15:54:32  dhardy
% edit presentation style
%
% Revision 1.6  1997/06/23 18:29:39  nealk
% Updated setenv stuff.
%
% Revision 1.5  1997/04/24 15:10:11  nealk
% Added items for machine-specific environment settings (setenv).
%
% Revision 1.4  1996/05/15 19:21:07  jean
% ready (I hope) for 1.4 beta release
%
% Revision 1.3  1995/06/30 15:51:43  brunner
% More complete.  Now includes sections for everything except Exemplar
% and Charm++.  Still needs proofreading.
%
% Revision 1.2  1995/06/29  16:39:04  brunner
% Single node and PVM instructions complete.  The rest still need work.
%
% Revision 1.1  95/06/19  16:08:17  16:08:17  nelson (Mark T. Nelson)
% Initial revision
% 
%%%%%%%%%%%%%%%%%%%%%%%%%%%%%%%%%%%%%%%%%%%%%%%%%%%%%%%%%%%%%%%%%%%%%%%%%%%%

\section{Running \NAMD}
\label{section:run}

\NAMD\ currently runs on network of HP, Sun, SGI, and Linux workstations, as
well as Cray T3E, IBM SP3, and SGI-CRAY Origin 2000 machines. On network
of workstations, a ``host program'' (called {\tt conv-host}) is needed to
launch \NAMD\ on individual workstations in the network. This host program
is included with the distribution. The general command line syntax for
invoking \NAMD\ on each of these platforms is:\\
\noindent {\tt <hostprog> namd2 <config-file> +p{\em N}}\\
Where {\tt <config-file>} is the name of the configuration file. 
And {\tt{\em N}} is the number of processors to be used for execution.

\subsection{Platform-Specific Notes}

The following subsections explain in detail the steps involved in running
\NAMD\ on particular platforms.

\subsubsection{Network of Workstations}

In order to run \NAMD\ on network of workstations, a host program (called
{\tt conv-host}) is needed in order to launch \NAMD\ on individual 
workstations. This program is bundled with the binary distribution.
The names of individual workstations as well as other control information
such as how many processes to create on those workstations (should they
hapen to be multiprocessor workstations) is specified to the host program
through a file named {\tt .nodelist} in the user's home directory or can
be superceded by the {\tt nodelist} file in the current directory. Details
about the syntax of {\tt nodelist} file can be found in {\em Converse
Installation and Usage Manual} at {\tt http://charm.cs.uiuc.edu/}.

\subsubsection{IBM SP3}

On IBM SP3, you would have to use the scheduler available on the front end
of the SP system in order to run \NAMD. Below, we give the description of
the commands we use at Argonne National Laboratory's IBM SP3 installation.
Consult the system documentation or your local system administrator about the
details at your site. We use the command {\tt spsubmit} to submit \NAMD\
runs to the SP scheduler at Argonne National Laboratory.\\
\noindent {\tt spsubmit -np {\em N} -maxtime {\em mins} namd2 <config-file>}\\
{\em N} is the number of processors we request \NAMD\ to run on, and
{\em mins} i the maximum time we expect \NAMD\ to take during this run.

\subsubsection{CRAY T3E}

{\tt mpprun} is the name of the command used to run parallel jobs on CRAY
T3E. This command can be used as:\\
\noindent {\tt mpprun namd2 <config-file> +p{\em N}}\\
where {\em N} is the number of requested processors. However, this can be
used only when the number of requested processors is less than or equal
to 32. For jobs larger than this, one has to use queuing facility provided
at the site of installation. Description of the queuing facilities is out
of this document's scope, and we leave that up to you to explore.

\subsubsection{Origin2000}

Origin2000 version of \NAMD\ is a shared memory version, and does not need
any particular {\em host} program to launch \NAMD. Thus, in order to run
\NAMD\ on Origin2000, we use the following command:\\
{\tt namd2 <config-file> +p{\em N}}\\
where, {\em N} is the number of processes we plan to use. However, note that
depending on the load and the queuing strategy employed by the system some of
the processes may be interleaved on some processors, thus causing the
performance to suffer. Some installations of Origin2000 employ an external
queuing facility to avoid this problem. You are strongly encouraged to
consult your local documentation for queuing facility used.

\subsection{Interactive modeling with MDScope}

Interactive molecular modeling can be performed using the MDScope
environment.  MDScope consists of \NAMD; \VMD, a program for
interactive visualization of biopolymers; and MDComm, a communications
library for efficient network transfer of molecular dynamics
information.  \VMD\ allows the user to view intermediate results of the
simulation being performed by \NAMD\ while the simulation is still 
proceeding.  For more information regarding 
the installation or use of MDScope, 
consult the \VMD\ documentation.  

