%%%%%%%%%%%%%%%%%%%%%%%%%%%%%%%%%%%%%%%%%%%%%%%%%%%%%%%%%%%%%%%%%%%%%%%%%%%%
%                                                                          %
%              (C) Copyright 1995 The Board of Trustees of the             %
%                          University of Illinois                          %
%                           All Rights Reserved                            %
%								  	   %
%%%%%%%%%%%%%%%%%%%%%%%%%%%%%%%%%%%%%%%%%%%%%%%%%%%%%%%%%%%%%%%%%%%%%%%%%%%%

%%%%%%%%%%%%%%%%%%%%%%%%%%%%%%%%%%%%%%%%%%%%%%%%%%%%%%%%%%%%%%%%%%%%%%%%%%%%
% RCS INFORMATION:
%
%       $RCSfile: ug_runit.tex,v $
%       $Author: jim $        $Locker:  $                $State: Exp $
%       $Revision: 1.6 $      $Date: 2003/08/07 15:19:06 $
%
%%%%%%%%%%%%%%%%%%%%%%%%%%%%%%%%%%%%%%%%%%%%%%%%%%%%%%%%%%%%%%%%%%%%%%%%%%%%
% DESCRIPTION:
%	This file contains the section of the NAMD User's Guide which
% describes how to actually run the program.
%
%%%%%%%%%%%%%%%%%%%%%%%%%%%%%%%%%%%%%%%%%%%%%%%%%%%%%%%%%%%%%%%%%%%%%%%%%%%%
% REVISION HISTORY:
%
% $Log: ug_runit.tex,v $
% Revision 1.6  2003/08/07 15:19:06  jim
% Fixed missing namd2 in command.
%
% Revision 1.5  2001/02/23 00:52:00  jim
% Many updates and cleanups for 2.3.
%
% Revision 1.4  2000/09/29 20:41:15  jim
% Updated many things for 2.2.
%
% Revision 1.3  1999/03/18 02:16:49  jim
% Spelling fixes.
%
% Revision 1.2  1998/03/09 23:42:15  milind
% Rewrote the entire section removing the PVM parts, and adding the converse
% details.
%
% Revision 1.1  1998/01/05 21:12:34  dhardy
% user guide, first draft
%
% Revision 1.8  1997/08/07 17:28:20  dhardy
% minor revisions
%
% Revision 1.7  1997/08/07 15:54:32  dhardy
% edit presentation style
%
% Revision 1.6  1997/06/23 18:29:39  nealk
% Updated setenv stuff.
%
% Revision 1.5  1997/04/24 15:10:11  nealk
% Added items for machine-specific environment settings (setenv).
%
% Revision 1.4  1996/05/15 19:21:07  jean
% ready (I hope) for 1.4 beta release
%
% Revision 1.3  1995/06/30 15:51:43  brunner
% More complete.  Now includes sections for everything except Exemplar
% and Charm++.  Still needs proofreading.
%
% Revision 1.2  1995/06/29  16:39:04  brunner
% Single node and PVM instructions complete.  The rest still need work.
%
% Revision 1.1  95/06/19  16:08:17  16:08:17  nelson (Mark T. Nelson)
% Initial revision
% 
%%%%%%%%%%%%%%%%%%%%%%%%%%%%%%%%%%%%%%%%%%%%%%%%%%%%%%%%%%%%%%%%%%%%%%%%%%%%

\section{Running \NAMD}
\label{section:run}

\NAMD\ runs on a variety of platforms.  Details of running on each
specific platform are given below and in the release notes included
in every distribution.

\subsection{Individual Workstations}

Individual workstations use the same version of NAMD as workstation
networks, but running NAMD is much easier.  You may launch any number
of namd2 processes on the local machine (for best performance lauch
one process per processor) using the \verb#++local# option via:

\begin{verbatim}
  charmrun namd2 ++local +p<procs> <configfile>
\end{verbatim}

There is no longer any need to be able to \verb#rsh localhost# or to
create a nodelist file containing the single host localhost.

Intel and Alpha processors produce binary files (restart and DCD
files) which must be \verb#byte-swapped# to be read on other platforms.
NAMD and VMD now handle this conversion automatically for most files.

\subsection{Individual Windows Workstations}

NAMD may be run on a single Windows workstation via the command:

\begin{verbatim}
  charmrun namd2 ++local +p<procs> <configfile>
\end{verbatim}

For best performance, <procs> should be the number of processors in
your machine, and defaults to one if the \verb#+p# option is omitted.
However, the \verb#++local# option is required unless \verb#charmd# is running
and a nodelist file (containing only localhost) is present.

See below to run on multiple machines.

\subsection{Workstation Networks}

Workstation networks require two files, the namd2 executable and the
charmrun program.  The charmrun program starts namd2 on the desired
hosts, and handles console I/O for the node programs.

To specify what machines namd2 will run on, the user creates a file
called \verb#nodelist#.  Below is an example nodelist file:

\begin{verbatim}
group main
 host brutus
 host romeo
\end{verbatim}

The \verb#group main# line defines the default machine list.  Hosts brutus
and romeo are the two machines on which to run the simulation.  Note
that charmrun may run on one of those machines, or charmrun may run
on a third machine.

The \verb#rsh# command (\verb#remsh# on HPUX) is used to start namd2 on each node
specified in the nodelist file.  If NAMD fails without printing any
output, check to make sure that \verb#rsh# works on your machine, by seeing
if \verb#rsh hostname ls# works for each host in the nodelist.  If you want
or need to use \verb#ssh# instead, then add \verb#setenv CONV_RSH ssh# to your
login or batch script and try \verb#ssh hostname ls# to each host first to
ensure that the machine is in your \verb#.ssh/known_hosts# file.  If you are
unable to use rsh or ssh, then add \verb#setenv CONV_DAEMON# and run \verb#charmd#
(or \verb#charmd_faceless#, which produces a log file) on every node.

Some automounters use a temporary mount directory which is prepended
to the path returned by the pwd command.  To run on multiple machines
you must add a \verb#++pathfix# option to your nodelist file.  For example:

\begin{verbatim}
group main ++pathfix /tmp_mnt /
 host alpha1
 host alpha2
\end{verbatim}

A number of parameters may be passed to charmrun.  The most important
is the \verb#+pX# option, where X specifies the number of processors.  If X
is less than the number of hosts in the nodelist, machines are
selected from top to bottom.  If X is greater than the number of
hosts, charmrun will start multiple processes on the machines,
starting from the top.  To run multiple processes on members of a SMP
workstation cluster, you may either just use the +p option to go
through the list the right number of times, or list each machine
several times, once for each processor.  The default is +p1.

You may specify the nodelist file with the \verb#++nodelist# option and the
group (which defaults to \verb#main#) with the \verb#++nodegroup# option.  If
you do not use \verb#++nodelist# charmrun will first look for \verb#nodelist#
in your current directory and then \verb#.nodelist# in your home directory.

If you always want to run on the machine you are logged in to you may
use \verb#localhost# in place of the hostname in your nodelist file, but
only if there are no other machines.  You will not need \verb#++pathfix#.
For example, \verb#.nodelist# in your home directory could read:

\begin{verbatim}
group main
 host localhost
\end{verbatim}

It is simpler in many cases to instead use the \verb#++local# option as
described under \verb#Individual Workstations# above, which eliminates the
need for the nodelist file and rsh entirely.

Once the nodelist file is set up, and you have your configuration file
prepared, run NAMD as follows:

\begin{verbatim}
  charmrun +p<procs> namd2 <configfile>
\end{verbatim}

\subsection{Windows Workstation Networks}

Windows is the same as other workstation networks described above,
except that rsh is not available on this platform.  Instead, you must
run the provided daemon (\verb#charmd.exe#) on every node listed in the
nodelist file.  Using \verb#charmd_faceless# rather than \verb#charmd# will eliminate
consoles for the daemon and node processes.  The \verb#++local# option is
also available under Windows, eliminating the need for \verb#charmd# and
nodelist when running NAMD only on the local machine.

\subsection{Scyld Beowulf Clusters}

Scyld Beowulf clusters replace rsh and other methods of launching jobs
via a distributed process space.  There is no need for a nodelist file
or any special daemons.  In order to allow access to files, the first
NAMD process must be on the master node of the cluster.  Launch jobs
from the master node of the cluster via the command:

\begin{verbatim}
  charmrun namd2 +p<procs> <configfile>
\end{verbatim}

For best performance, run a single NAMD job on all available nodes and
never run multiple NAMD jobs at the same time.  You may safely suspend
and resume a running NAMD job on a Scyld Beowulf using control-Z or
\verb#kill -STOP# and \verb#kill -CONT# on the charmrun process.

\subsection{Compaq AlphaServer SC}

Although NAMD uses MPI and the Elan library on this platform, parallel
jobs are run using the \verb#prun# command.  (The standard MPI charmrun is
wrong on this platform.)  The syntax for this command is:

\begin{verbatim}
  prun -n <procs> namd2 <configfile>
\end{verbatim}

There are additional options.  Consult your local documentation.

\subsection{IBM RS/6000 SP}

Run NAMD as you would any POE program.  The options and environment
variables for poe are various and arcane, so you should consult your
local documentation for recommended settings.  As an example, to run
on Blue Horizon one would specify:

\begin{verbatim}
  poe namd2 <configfile> -nodes <procs/8> -tasks_per_node 8
\end{verbatim}

\subsection{Cray T3E}

The T3E version has been tested on the Pittsburgh Supercomputer Center
T3E.  To run on <procs> processors, use the mpprun command:

\begin{verbatim}
  mpprun -n <procs> namd2 <configfile>
\end{verbatim}

\subsection{Origin 2000}

For small numbers of processors (1-8) use the non-MPI version of namd2.
If your stack size limit is unlimited, which DQS may do, you will need
to set it with \verb#limit stacksize 64M# to run on multiple processors.
To run on <procs> processors call the binary directly with the +p option:

\begin{verbatim}
  namd2 +p<procs> <configfile>
\end{verbatim}

For better performance on larger numbers of processors we recommend
that you use the MPI version of NAMD.  To run this version, you must
have MPI installed.  Furthermore, you must set two environment
variables to tell MPI how to allocate certain internal buffers.  Put
the following commands in your .cshrc or .profile file, or in your
job file if you are running under a queuing system:

\begin{verbatim}
  setenv MPI_REQUEST_MAX 10240
  setenv MPI_TYPE_MAX 10240
\end{verbatim}

Then run NAMD with the following command:

\begin{verbatim}
  mpirun -np <procs> namd2 <configfile>
\end{verbatim}


